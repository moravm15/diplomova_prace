% options:
% thesis=B bachelor's thesis
% thesis=M master's thesis
% czech thesis in Czech language
% slovak thesis in Slovak language
% english thesis in English language
% hidelinks remove colour boxes around hyperlinks

\documentclass[thesis=M,czech]{FITthesis}[2012/06/26]

\usepackage[utf8]{inputenc} % LaTeX source encoded as UTF-8

\usepackage{float}

\usepackage{listings}
\usepackage{algorithm}
\usepackage{algorithmic}
\usepackage{listings}
\usepackage{algorithm}
\usepackage{algorithmic}

\usepackage{graphicx} %graphics files inclusion
% \usepackage{amsmath} %advanced maths
% \usepackage{amssymb} %additional math symbols

\usepackage{dirtree} %directory tree visualisation
\usepackage{listings}

\lstdefinelanguage{VHDL}{
	morekeywords={
		CALL,FUNCTION,DESTINATION,EXPORTING,IMPORTING
	}
}

\lstdefinelanguage{java}{
	morekeywords={
	function, return, if, this, that
	}
}


% % list of acronyms
% \usepackage[acronym,nonumberlist,toc,numberedsection=autolabel]{glossaries}
% \iflanguage{czech}{\renewcommand*{\acronymname}{Seznam pou{\v z}it{\' y}ch zkratek}}{}
% \makeglossaries

\newcommand{\tg}{\mathop{\mathrm{tg}}} %cesky tangens
\newcommand{\cotg}{\mathop{\mathrm{cotg}}} %cesky cotangens

% % % % % % % % % % % % % % % % % % % % % % % % % % % % % % 
% ODTUD DAL VSE ZMENTE
% % % % % % % % % % % % % % % % % % % % % % % % % % % % % % 

\department{Katedra softwarového inženýrství}
\title{Vývoj FIORI aplikace nad SAP PM modulem pro realizaci servisních zakázek a preventivní údržby}
\authorGN{Marcel} %(křestní) jméno (jména) autora
\authorFN{Morávek} %příjmení autora
\authorWithDegrees{Bc. Marcel Morávek} %jméno autora včetně současných akademických titulů
\author{Marcel Morávek} %jméno autora bez akademických titulů
\supervisor{Ing. Martin Šindlář}
\acknowledgements{Poděkování}
\abstractCS{Abstrakt CZ}
\abstractEN{Abstrakt EN}
\placeForDeclarationOfAuthenticity{V~Praze}
\declarationOfAuthenticityOption{4} %volba Prohlášení (číslo 1-6)
\keywordsCS{SAP, Fiori}
\keywordsEN{SAP, Fiori}

% \website{http://site.example/thesis} %volitelná URL práce, objeví se v tiráži - úplně odstraňte, nemáte-li URL práce

\setcounter{secnumdepth}{5}

\begin{document}

% \newacronym{CVUT}{{\v C}VUT}{{\v C}esk{\' e} vysok{\' e} u{\v c}en{\' i} technick{\' e} v Praze}
% \newacronym{FIT}{FIT}{Fakulta informa{\v c}n{\' i}ch technologi{\' i}}

\begin{introduction}
	Tato práce se zaobírá ...
\end{introduction}

\chapter{Cíl práce}
Cílem této práce je vytvoření webové SAP Fiori aplikace nad SAPovským modulem údržby ve frameworku SAPUI5. Pomocí této aplikace bude umožněno realizovat servisní zakázky i preventivní údržbu strojů a to včetně jejich vybavení.

\section{Vývojová část}
Cílem praktické části je navržení uživatelského rozhraní aplikace s ohledem na způsob zacházení s modulem údržby. Nadále pak implementace samotné aplikace dle provedeného návrhu. 

\section{Rešeršní část}
Jedním z cílů rešeršní části je porovnání prostředí podporujících vývoj ve frameworku SAPUI5. 

\section{Co není cílem práce}
Cílem této práce není implementace ani návrh funkčnosti uvnitř EPRového systému. Tato práce začíná na úrovni komunikačních rozhraní jednotlivých funkčních modulů realizujících požadované operace. 

\chapter{SAP}
Kapitola obecně popisuje podnikový informační systém SAP. V jednotlivých podkapitolách jsou pak popsány informace o historii firmy a architektonické struktuře systému. Dále jsou zde popsány i jednotlivé technické komponenty, které jsou použity pro realizaci požadované aplikace. 

\section{Společnost SAP}
Společnost SAP je v současné době jedním z největších poskytovatelů podnikových aplikací a jednou z největších softwarových společností na celém světe. Pod zkratkou SAP se schovávají počáteční písmena německých slov „Systeme, Anwendungen, Produkte in der Datenverarbeitung“. Anglicky si lze zkratku přeložit pomocí anglických slov „Systems - Applications - Products in data processing“. Zaměřují se na vývoj a provoz softwarových produktů podporujících podnikové procesy. Zejména pak tedy řízení podniku, systémy pro správu vztahu se zákazníky a v současné době pak především vývoj technologií pro webové aplikace a cloud computing \cite{sap_information}.

\section{SAP R3}
První verze systému SAP R/1, tvořená pouze finančním účetnictvím, byla vydána již v roce 1973. Následující verze SAP R/2, vydaná o šest let později, se dá již označit za první funkční ERP systém (Enterprise resources planning). Ovšem nevýhodou tohoto systému byla vysoká technická náročnost na klienta, která v tu dobu přinášela nutnost využívání sálových počítačů. Verzí SAP R/3 z roku 1992 však byla kompletně změněna architektura SAPu. Změnou architektury na klient-server opadly nároky na klienta a odpadla tak tehdejší nutnost využívání sálových počítačů a zároveň se začaly využívat relační databáze. Hlavní výhoda této architektury spočívala především však pak v kompatibilitě s různými platformami a operačními systémy Microsoft Windows nebo Unix. Tím se společnost dostala na špici poskytovatelů ERP systémů a na té se do dnes drží.

\paragraph{Architektura systému SAP R/3} Spolu se změnou architektury s příchodem verze SAP R/3 na model klient-server, došlo k rozčlenění do tří vrstev:
\begin{itemize}
	\item
	\textbf{Databázová vrstva} je tvořena vlastními databázovými servery, které slouží pro ukládání dat. Jelikož SAP je multiplatformní systém, vývojáře nemusí zajímat, na jaké platformě (UNIX, ORACLE, SUN, MICROSOFT nebo jiné) databázová vrstva běží, protože na aplikační vrstvě bude přístup vypadat vždy stejně. 
	
	Příkladem mohou být uložená data týkající se vybavení továrny, která jsou roztroušená po jednotlivých databázových tabulkách.
	\item
	\textbf{Aplikační vrstva} slouží především jako prostředí pro vykonávání programové logiky na straně aplikačního serveru. Centrálně se na něm zpracovávají data, které se načítají a ukládají z databázové vrstvy. Jednotlivé programy (funkce) se v tomto prostředí píší zpravidla v SAPovském programovacím jazyce ABAP, někdy však i pomocí Javy. 
	
	Příkladem budiž uživatelský požadavek o vyčtení zakázek na výrobní stroj pro daného uživatele. Aplikační vrstva nejdříve musí načíst z databázové vrstvy potřebná data o uživateli, strojích v továrně i zakázkách a ty poté následně dle požadavku zpracovat. Vyloučí tak například stroje, ke kterým uživatel nemá oprávnění nebo zakázky nerelevantní k danému času a strojům. Tím získá požadovanou informaci, se kterou je poté následně nějakým způsobem naloženo, například přenosem do vyšší, prezentační vrstvy.
	\item
	\textbf{Prezenční vrstva} slouží pro komunikaci mezi uživatelem a počítačem. Má za úkol předávat informace uživateli. Vlastní komunikace probíhá na prezentačním serveru, tedy klientské části. Její nedílnou součástí je rozhraní SAP GUI, které se stará o komunikaci mezi prezentačním a aplikačním serverem. To je ovšem v poslední době nahrazováno přístupem přes webový prohlížeč. Typickým příkladem jsou SAPovské aplikace Fiori vytvořené ve frameworku SAPUI5, kterým je věnována samostatná sekce. 
\end{itemize} 	

\subsection{Moduly SAP R3}
Systém SAP R/3 je vnitřně rozdělen do několika různých modulů. Každý z nich pak řeší konkrétní problematiku firmy.

\begin{figure}[H]
	\centering
	\includegraphics[width=1\textwidth]{images/sap_r3.jpg}
	\caption{Moduly SAP R3}
	\label{img:sapr3}
	\small
	fghfghgddh
\end{figure}

\begin{itemize}
	\item
	\textbf{Financial Accounting (FI)} označuje finanční účetnictví a je jedním z nejdůležitějších modulů SAP ERP. Používá se k uložení finančních dat organizace a pomáhá analyzovat finanční podmínky společnosti na trhu.
	\item
	\textbf{Controlling (CO)} podporuje koordinaci, monitorování a optimalizaci všech procesů v organizaci. Zahrnuje správu a konfiguraci základních dat, které pokrývají náklady a výnosy, interní objednávky a další nákladové prvky a funkční oblasti. Jeho hlavním účelem je plánování. Umožňuje určit odchylky srovnáním skutečných dat s údaji plánu a tím umožňuje řídit obchodní toky v organizaci.
	\item
	\textbf{Asset Management (AM)} slouží k optimální správě fyzického majetku organizace. Zahrnuje takové funkcionality jako jsou návrh, konstrukce, provoz, údržba a výměna zařízení. Spravuje majetek v jednotlivých odděleních (obchodních jednotkách).
	\item
	\textbf{Project system (PS)} je nástroj pro správu dlouhodobých projektů. Umožňuje uživatelům plánovat finanční prostředky i zdroje a kontrolovat jednotlivé části projektu tak, aby bylo zaručeno včasné dodání pokud možno v rámci rozpočtu.
	\item
	\textbf{Workflow (WF)} umožňuje navrhovat a realizovat obchodní procesy v rámci aplikačních systémů SAP. Zajišťuje aby se práce dostala v požadovaný čas do rukou správným lidem. Jeho cílem je usnadnění automatizace podnikových procesů.
	\item
	\textbf{Industry Solutions (IS)} poskytuje specifická řešení pro desítky industriálních odvětví jako například pro automobilový, chemický či energetický průmysl.
	\item
	\textbf{Human Resources (HR)} umožňuje organizaci strukturálně a efektivně zpracovávat informace údaje týkající se zaměstnanců k potřebám obchodním požadavkům.
	\item
	\textbf{Plant Maintenance (PM)} poskytuje nástroj pro provádění veškerých potřebných činností týkajících se údržby organizace a jejích součástí. Umožňuje plánovat údržbu i s ohledem na materiálovou potřebu, zaznamenávat a vyrovnávat náklady spojené s činností.
	\item
	\textbf{Materials Management (MM)} se zabývá řízením materiálů a skladových zásob. Kontroluje, aby nedocházelo k nedostatkům zboží a nevznikaly tak mezery v řetězci dodavatelského procesu.
	\item
	\textbf{Production Planning (PP)} sleduje a zaznamenává toky ve výrobním procesu. Má za úkol sladění poptávky s výrovním kapacitou spolu s vytvořením plánů k dokončení komponentů a produktů.
	\item
	\textbf{Quality Management (QM)} je modul úzce provázaný s moduly MM, PP či PM a nedílnou součástí logistického řízení. Používá se k prováděnější kvalitativních funkcí jako je plánovaní jakosti, zajištění a kontroly kvality ve výrobním a spotřebním procesu.
	\item
	\textbf{Sales and Distribution (SD)} se používá pro ukládání údajů o zákaznících a produktech organizace. Pomáhá řídit fakturaci, prodej a přepravu produktů či služeb organizace. Řídí vztah se zákazníky od počáteční nabídky až po prodejní zakázku a fakturaci produktu.   
\end{itemize} 	

\section{SAP Plant Maintenance (PM)}
Modul SAP Plant Maintenance je komplexní řešení, které poskytuje nástroje pro kompletní údržbu v rámci organizace. Veškeré prováděné činnosti jsou vzájemně propojeny s návaznými moduly (Production Planning, Materials Management a Sales and Distribution) v rámci podnikových procesů. Modul umožňuje provádět komplexní plánování, realizovat denní činnosti údržby nebo zaznamenávat případné vzniklé problémy. Díky provázanosti na ostatní moduly taktéž dovede sledovat a plánovat materiálové aktivity případně zaznamenávat i určovat dané náklady vzniklé údržbou.

K realizování zmíněných aktivit je modul rozdělen do následujících podmodulů:
\begin{itemize}
	\item
	Správa technických objektů a vybavení
	\item
	Plánování úkolů údržby
	\item
	Řízení notifikací v rámci nastavených procesů a zakázek v rámci údržby
\end{itemize} 	

\subsection{Technické objekty}
Jelikož je v organizaci zapotřebí správně a efektivně spravovat jednotlivé aktivity v rámci procesů modulu Plant Maintenance, je struktura údržby rozdělena na technické objekty. Ty slouží k definování jednotlivých typů strojů, kterými organizace disponuje. Za použití charakteristiky technických objektů lze pak zadefinovat jiný technický objekt, což umožňuje hierarchicky definovat strukturu organizace. Výčtem zmíněných vlastností technických objektů získáme následující výhody.
\begin{itemize}
	\item
	Doba potřebná pro správu jednotlivých technických objektů je snížena.
	\item
	Zpracování údržby je zjednodušeno.
	\item
	Doba strávená při zadávání dat během zpracování údržby je značně snížena.
	\item
	Konkrétnější, důkladnější a rychlejší vyhodnocení údajů o údržbě.
\end{itemize}
Technická správa objektů se skládá z následujících činností:
\begin{itemize}
	\item
	\textbf{Inspekce} - měřit a sledovat aktuální stav technického objektu
	\item
	\textbf{Preventivní údržba} - předvídat potřebu oprav a udržovat optimální stav technického objektu
	\item
	\textbf{Oprava} - měření a obnovení technického objektu
	\item
	\textbf{Další činnosti související s údržbou}
\end{itemize}

\vspace*{0.5cm}
Zpracování údržby pomáhá řídit skutečné údržbářské práce prováděné v údržbě. Proces se skládá ze tří oblastí:
\begin{itemize}
	\item
	\textbf{Upozornění na údržbu} -  oznamte poruchu nebo popište technickou podmínku objektu
	\item
	\textbf{Objednávka údržby} - provést podrobný plán údržby a sledovat průběh práce a uhradit náklady na údržbu
	\item
	\textbf{Historie údržby} - uložení důležitých údajů údržby pro vykazování a vyhodnocení
\end{itemize} 	

\subsection{Preventivní údržba}
Preventivní údržba je dlouhodobý proces, jehož cílem je zajistit vysokou použitelnost zařízení a funkčních míst a minimalizovat prostoje způsobené opravami. Tato funkce podporuje údržbu založenou na výkonu, pokud jsou měřicí body nebo čítače používány pro řízení technických podmínek objektu.
Součást preventivní údržby lze použít k:
\begin{itemize}
	\item
	Uložit seznam úkolů, které mají být provedeny
	\item
	Upřesněte rozsah inspekčních prací, preventivní údržbu a plánování činností
	\item
	Zadejte opakovanou frekvenci údržby
	\item
	Upřesněte přiřazení kontrolních činností a preventivní údržbu na základě nákladů
	\item
	Vyhodnotit náklady na budoucí preventivní údržbu a inspekční práci
\end{itemize} 	

Preventivní údržba v organizaci se používá k zabránění selhání systému a rozpadu výroby. Pomocí preventivní údržby můžete ve vaší organizaci dosáhnout různých výhod. Preventivní údržba se používá k provádění inspekcí, preventivní údržby a oprav. Plány údržby slouží k definování dat a rozsahu úkolů preventivní a inspekční údržby, které lze naplánovat pro technické objekty.

Seznam úkolů v Preventivní údržbě je definován jako sled činností, které jsou prováděny v rámci preventivní údržby v organizaci. Jsou používány k provádění opakovaných úkolů v rámci preventivní údržby a k jejich efektivnímu provedení.

Pomocí seznamů úkolů můžete snížit úsilí standardizací pracovní postup. Všechny aktualizace se provádějí na jednom konkrétním místě v seznamu úkolů údržby a všechny položky údržby a údržby v systému obdrží aktualizovaný stav pracovních postupů. Pomocí seznamů úkolů pomáhá při snižování úsilí potřebného pro vytvoření objednávek údržby a položek údržby, jak můžete vrátit do seznamu úkolů, abyste viděli pracovní postup.
Klíčové funkce seznamů úkolů v SAP Plant Maintenance jsou následující plánovaná a probíhající údržba podrobněji popsány v následujících odstavcích.

\paragraph{Plánovaná údržba}
Všechny plánované činnosti, jako je kontrola, údržba a opravy, jsou součástí plánované údržby. V údržbě rostlin definujete časové intervaly, kdy je třeba pracovní kroky provést a pracovní sekvence, ve kterých musí být provedeny. Seznamy úkolů jsou při plánování plánování údržby přiřazeny plánu údržby.

\paragraph{Probíhající údržba}
Seznam úkolů pro průběžnou údržbu obsahuje pracovní postupy založené na aktuální kontrole. Všechna kontrola, která se provádí bez pravidelného rozvrhu, je předmětem trvalé údržby.

\subsection{Zpracování údržby}
Zpracování údržby se skládá z několika úrovní, které nemusí být nutně plně realizovány.

Proto je možné zpracovat opravu v mnoha fázích plánování, jako je předběžná kalkulace, plánování práce, materiálové zabezpečení, plánování zdrojů a povolení. Je však také možné okamžitě reagovat na škody způsobené událostmi, které způsobí vypnutí výroby, a v co nejkratší možné době předložit požadované objednávky a prodejní doklady s minimálními údaji.

Zpracování údržby lze rozdělit na následující tři oblasti:
\begin{itemize}
	\item
	\textbf{Popis stavu objektu} - Nejdůležitějším prvkem v této oblasti je oznámení o údržbě. Používá se k popisu stavu technického objektu nebo hlášení poruchy na technickém objektu a požadavek na opravu poškození.
	\item
	\textbf{Provádění úkolů údržby} - Nejdůležitějším prvkem v této oblasti je objednávka údržby. Používá se k detailnímu plánování provádění údržbářských činností, sledování průběhu práce a vypořádání nákladů na údržbu.
	\item
	\textbf{Dokončení úkolů údržby} - Nejdůležitějším prvkem v této oblasti je historie údržby. Používá se k dlouhodobému uložení nejdůležitějších údajů o údržbě. Tyto údaje lze kdykoli vyžádat k vyhodnocení.
\end{itemize} 

Tyto prvky umožňují zpracovat všechny úkoly, které je třeba provést v údržbě zařízení, stejně jako operace, které nepatří přímo do údržby zařízení, jako jsou investice, restrukturalizace, úpravy a podobně.

\begin{figure}[H]
	\centering
	\includegraphics[width=1\textwidth]{images/pm_process.jpg}
	\caption{Proces diagram PM}
	\label{img:pm_process}
	\small
	Procesní diagram PM
\end{figure}

\section{SAP BSP}
Tato sekce se zaměřuje na front-endovou technologii SAP BSP. Jedná se o jednu z technologií použitých v cílové architektuře řešení mobilní aplikace a proto je zde stručně popsána její struktura.

\paragraph{SAP Web Application Server} SAP WAS je produktový produkt aplikačního serveru a nová generace produktu Basis. Poskytuje všechny funkce, které Basis udělal, a pak mnohem víc. Přemýšlejte o tom jako o nadpřirozené základně. Přemýšlejte o tom jako o obalení základny s obrovskými možnostmi webových aplikací, mezi které patří i schopnost spouštět aplikace Java / J2EE vedle aplikací ABAP. A chci říci, že přidání Java / J2EE k tomuto aplikačnímu serveru v žádném případě neohrožuje podporu pro ABAP. Všechny vaše investice do řešení ABAP jsou dobře chráněny. Rozdíl spočívá v tom, že oba vývojové a běhové prostředí ABAP a Java / J2EE se nacházejí na jedné platformě, na jedné společné infrastruktuře. Tato sjednocená oblast systému ABAP / Java minimalizuje úsilí učitele o výuku a náklady na správu.

\paragraph{BSP} Business Server Page (BSP) je kompletní funkční aplikace, stejně tak jako klasická transakce SAP. Rozhraním pro přístup však není software SAPGUI, spíše však libovolný webový prohlížeč. Díky využití protokolů HTTP nebo HTTPS je protokol používaný pro přístup k aplikaci po celé síti, což umožňuje používání standardních produktů, jako jsou firewally a proxy servery.

Programovací program Stránky Business Server je podobný technologii serverových stránek. Zaměřením programovacího modelu BSP jsou body, které zajišťují optimální strukturu v rozhraní a obchodní logiku.

\begin{figure}[H]
	\centering
	\includegraphics[width=1\textwidth]{images/bsp.png}
	\caption{Struktura BSP aplikace}
	\label{img:bsp_structure}
	\small
	Struktura BSP aplikace
\end{figure}

Každá BSP stránka využívá stejného principu zpracování dat. Obecně lze na BSP stránce definovat šest základních událostí, na které se zpracovávají data. 
\begin{itemize}
	\item
	\textbf{OnCreate}:
	\item
	\textbf{OnRequest}:
	\item
	\textbf{OnInitialization}:
	\item
	\textbf{OnManipulation}:
	\item
	\textbf{OnInputOutputProcessing}:
	\item
	\textbf{OnDestroy}:
\end{itemize} 

\section{SAP FIORI}

Stejně jako každý jiný tradiční počítačový software byly také přístupné ERP systémy prostřednictvím grafického uživatelského rozhraní na stacionárních místech PC nebo notebooků. Vzhledem k rychlému vývoji mobilních zařízení očekává velké množství uživatelů zlepšení a příležitosti pro mobilní použití.

V minulosti, mnoho zákazníků SAP vyjádřili svou nespokojenost ohledně staromódní vzhled a dojem z obrazovek SAP, stejně jako nedostatek exkluzivního přístupu přes desktop GUI pro většinu operací (jako schvalování objednávky, vytvoření prodejní objednávky, samostatně výdělečně servisní úkoly, vyhledávání informací.) Zpětná vazba byla oceněna a SAP podnikla kroky ke zlepšení použitelnosti a dostupnosti. (Bince 2015, 365)
Dne 15. května 2013 představila společnost SAP platformu SAP Mobile Platform 3.0, otevřenou platformu, která byla k dispozici vývojářům softwaru. Společně se zavedením platformy zahájila společnost SAP svůj nový mobilní produkt s názvem Fiori. (SAP Fiori 2013, citováno 9. listopadu 2016.) Tento nový produkt je založen na pěti zásadách návrhu (obrázek 4).
Prostřednictvím tohoto nového mobilního řešení uživatelsky orientované klienti nyní měl přístup k novým řešením, kde sbírka aplikací bylo možné použít na různých zařízení, jako jsou stolní počítače, chytré telefony a tablety. V prvním vydání Flowers bylo zařazeno 25 aplikací, které slouží klientům v jejich nejčastějších obchodních funkcích. (SAP Fiori 2013, citováno 9. listopadu 2016.)


\begin{figure}[H]
	\centering
	\includegraphics[width=1\textwidth]{images/fiori.jpg}
	\caption{Struktura BSP aplikace}
	\label{img:fiori}
	\small
	Struktura BSP aplikace
\end{figure}


\begin{itemize}
	\item
	Založené na rolích:
	- Uživatelsky orientované aplikace závislé na odpovědnosti uživatele
	- uživatel může mít více rolí a spouštět různé úkoly v několika doménách
	\item
	citlivý:
	- založené na formátu HTML5; pracuje bez problémů na různých zařízeních a velikostech obrazovky
	- automaticky upravuje rozložení aplikací na dostupné obrazovce
	- podporuje různé režimy interakce, jako klávesnice, myši a dotykové vstupy
	\item
	Jednoduché:
	- Jednoduché uživatelské rozhraní podporuje rychlé a snadné dokončení úkolů
	- má přístup 1: 1: 3: jeden uživatel, jeden případ, tři obrazovky (stolní, tabletové, mobilní)
	\item
	koherentní:
	- uživatel může mít mnoho aplikací, které mají stejný design a použitelnost
	- Snadno se naučíte nové aplikace poté, co se učíte používat jednu aplikaci Fiori
	\item
	Okamžitá hodnota:
	- stejný návrhový vzorec v aplikacích snižuje čas a náklady na školení nových uživatelů
\end{itemize} 	

\begin{figure}[H]
	\centering
	\includegraphics[width=1\textwidth]{images/fiori_arch.png}
	\caption{Struktura BSP aplikace}
	\label{img:fiori_arch}
	\small{ https://sapui5.hana.ondemand.com/\#/topic/28b59ca857044a7890a22aec8cf1fee9.html }
\end{figure}

\subsection{MVC}

\begin{figure}[H]
	\centering
	\includegraphics[width=0.6\textwidth]{images/mvc.png}
	\caption{Struktura BSP aplikace}
	\label{img:bsp}
	\small
	Struktura BSP aplikace
\end{figure}

\paragraph{Model} reprezentuje správu vlastních dat, nad nimiž aplikace pracuje: to mů-že být třeba obrázek, textový dokument, databáze uložená na serveru SQL, … Je dobré si uvědomit rozdíl mezi daty samotnými a jejich správcem (tedy prá-vě tím modelem, o němž zde hovoříme). Třeba takový konkrétní obrázek jsou data; skupina tříd, umožňující obrázky načítat ze souborů a opět do nich uklá-dat, zjišťovat a měnit jejich atributy, převádět jejich formáty a pracovat s jejich obsahem – to je správce. V běžící aplikaci pak samozřejmě bude základním ob-jektem vrstvy modelu nějaká vhodná instance, jež bude reprezentovat vlastní obrázek a nabízet všechny odpovídající služby.

\paragraph{View} zahrnuje všechny objekty (grafického) uživatelského rozhraní a jejich služby. Uživatel aplikace s ní komunikuje výhradně prostřednictvím těchto ob-jektů; právě ony mu prezentují data z modelu ve vhodné formě, a naopak pou-ze jejich prostřednictvím uživatel dává aplikaci příkazy, jež určují její další činnost.
View má s modelem společnou tu nejdůležitější věc: jeho objekty nejsou závislé na konkrétní aplikační logice. Grafické uživatelské rozhraní aplikace tak mů-žeme podle potřeby (a podle požadavků zákazníků) snadno kdykoliv měnit, aniž by bylo zapotřebí přitom nějak zasahovat do aplikační logiky (nebo do-konce do modelu).

\paragraph{Controller} mezi datovým modelem a objekty grafického uživatelského roz-hraní, jež tvoří vzhled, stojí vrstva controller; právě na její úrovni je imple-mentována funkční logika aplikace, takové věci jako „stiskne-li uživatel tohle tlačítko, provede se támhleta akce, a v tomto textovém poli se zobrazí výsle-dek“. (Čada, 2009)

\begin{figure}[H]
	\centering
	\includegraphics[width=1\textwidth]{images/fiori.jpg}
	\caption{Struktura BSP aplikace}
	\label{img:fiori}
	\small
	Struktura BSP aplikace
\end{figure}

\subsection{SAP ODATA}

\paragraph{Protokol OData} umožňuje vytváření datových služeb založených na webovém protokolu REST (representational state transfer), který umožňuje uživatelům provádět CRUDQ operace nad zdroji identifikovanými pomocí Uniform Resource Identifier (URI) a definovanými v datovém modelu použitím jednoduchých HTTP zpráv.

Protokol původně vyvinul Microsoft, verze 1.0, 2.0 a 3.0 jsou uvolněny pod Microsoft Open Specification Promise. Aktuálně nejnovější verze 4.0 byla schválena jako standard prostřednictvím OASIS OData Technical Committee, jejímiž členy jsou BlackBerry, IBM, Microsoft, SAP a další. Následující informace ohledně OData protokolu se budou vztahovat k verzi 2.0, která je momentálně v SAPu navzdory zavádění verze 4.0, používána nejčastěji. 

Data přenášená prostřednictvím OData protokolu mohou využívat různé datové formáty běžně používané ve webových technologiích, například Extensible Markup Language (XML), JavaScript Object Notation (JSON) nebo Atom Publishing Pro-tocol (AtomPub) a další. Data jsou při přenosu zabalena do protokolu HTTP, případně do jeho zabezpečené verze HTTPS.

Jádrem OData jsou feeds, které jsou kolekcemi Collections složené ze záznamů Entries. Každý Entry je identifikovaný klíčem a reprezentuje strukturovaný záznam, který má seznam vlastností Properties, ty mohou být komplexního, nebo primitivního typu. Entries mohou být součástí hierarchie typů a mohou mít související entries a feeds odkazované prostřednictvím Links.

\paragraph{Metadata} OData služba poskytuje metadata dokumenty. Aby mohli uživatelé oData služby prozkoumat, co všechno služba nabízí, aniž by museli zkoumat implementaci služ-by na backendu.
Jednodušší Service Document se nachází v root URI služby, obsahuje seznam všech feedů, takže je uživatelé mohou prozkoumat a zjistit jejich adresy.
Přidáním segmentu \$metadata do root URI služby získáme Service Metadata Document, který popisuje celý datový model, jinými slovy strukturu a propojení všech zdrojů. Jak uvádí (Odata, 2013), Service Metadata Document popisuje svá data pomocí termínů EDM použitím XML jazyka pro popis modelů nazvaných Con-ceptual Schema Definition Language (CSDL). Tento CSDL dokument je pak zabalen použitím formátu EDMX. (Odata, 2015)

\paragraph{Entity Data Model (EDM)} Hlavním konceptem v EDM jsou entity a asociace. Entity jsou instance Entity Type (například Inventura, Závod, Budova a tak dále), které jsou strukturovanými zá-znamy s klíčem Entity Key a složenými z pojmenovaných a typovaných vlastností. Entity Key je složen z podmnožiny vlastností v Entity Type. Entity Key (například InventuraID, nebo BudovaID) je zásadní koncept pro unikátní identifikaci instancí Entity Type a umožnění Enity Type instancím fungovat ve vztazích. Entity jsou se-skupovány v Entity Sets (například Budovy je množina instancí Entity Type Budo-va).
Asociace definují vztah mezi dvěma nebo více Enity Type (například Budova patří Závodu). Instance asociací jsou seskupovány v Association Sets. Navigation Properties jsou speciální vlastnosti v Entity Type, které jsou vázány na konkrétní asociaci a mohou být použity k odkazování na asociaci entity.
Položením předchozích definic do OData termínů, feedy vystavované OData službou jsou reprezentovány pomocí Entity Set, nebo Navigation Property na Entity Type, které identifikuje kolekci entit. Například Entity Set identifikovaný pomocí URI http://services.odata.org/OData/OData.svc/Products, nebo kolekce entit identifikovaná pomocí Products Navigation Property v http://services.odata.org/OData/OData.svc/Categories(1)/Products identifikují feed složený z Entry vystavovaný OData službou. (Odata, 2015)

\begin{figure}[H]
	\centering
	\includegraphics[width=1\textwidth]{images/odata.png}
	\caption{Struktura BSP aplikace}
	\label{img:bsp}
	\small
	Struktura BSP aplikace
\end{figure}

\chapter{Analýza a návrh aplikace}
Tato kapitola se věnuje analýze a návrhu požadované webové aplikace pracující nad SAP modulem Plant Maintenance pro vykonávání údržby v halách. Jednotlivé podkapitoly se pak věnují zpracování požadavků kladených na výslednou aplikaci, omezením aplikace a definování uživatelských rolí, které budou v aplikaci použity. Na základě stanovených požadavků je posléze proveden návrh uživatelského rozhraní.  


\section{Základní popis aplikace}
Webová aplikace bude svým uživatelům umožňovat vykonávat potřebné úkony pro správný chod výrobních hal. To zahrnuje důležité operace jako je například hlášení požadavků na údržbu nebo poruch pro zadefinované vybavení, evidování práce na nahlášených poruchách apod. Podrobný soupis funkčních požadavků je popsán v následující kapitole. 


\section{Uživatelské role}
\label{sec:role}
Uživatelské role popsané této sekci jsou vytvořeny na základě požadavků v kapitole níže. Ovšem pro lepší čitelnost jsou popsány zde, ještě před důvody jejich vzniku. 
\subsection{Údržbář}
\label{ssec:udrzbar}
Uživatel, který primárně řeší plánované údržby z preventivních důvodů (například periodicky nastavené v modulu PM), povinnosti vyřešit vznesený požadavek nebo nahlášené poruchy.
\subsection{Operátor výroby}
\label{ssec:operator_vyroby}
Uživatel, jehož primárním účelem je obsluha strojů na jeho pracovišti ve výrobním procesu. S aplikací přijde do styku pouze v případě, že bude chtít nahlásit poruchu na daném vybavení. Jednotlivá pracoviště jsou vybavena stolními počítači na kterých jsou spuštěny aplikace pro provoz. Z těchto aplikací bude umožněn odkázání se do budoucí webové aplikace pro nahlášení poruchy na příslušném pracovišti.
\subsection{Uživatel s možností založení požadavku na údržbu}
\label{ssec:pozadavek_udrzba}
Uživatel, jehož zodpovědností je bezproblémový chod strojů. Osoba by se dala charakterizovat jako revizní technik, který má na starost obcházení všech pracovišť a kontrolu jednotlivých výrobních linek. V případě, že shledá za vhodné provést na nějakém stroji údržbu, založí adekvátní požadavek. To bude zpravidla provádět z přenosného zařízení, které má neustále u sebe . Tím může být například chytrý telefon nebo tablet.
\subsection{Správce - administrátor}
\label{ssec:administrator}
Uživatel zodpovědný za správu ostatních uživatelských účtů. Pomocí rolí bude definovat možnosti jednotlivých uživatelů. Jelikož role nejsou uživatelsky výlučné, budou dvě fyzické osoby představující údržbáře mít odlišné možnosti v aplikaci. Oba dva budou moci provádět údržbářské činnosti, ale jenom jeden z nich bude moci zakládat požadavky na údržbu.


\section{Model požadavků}
V této sekci jsou uvedeny veškeré požadavky kladené na výslednou aplikaci, které byly probírány se zadavatelem. Požadavky představují minimální kritéria potřebná pro samotný návrh uživatelského rozhraní. Veškeré požadavky byly probírány se zadavatelem, většina z nich byla jasně stanovena v rámci rámcového zadání, některé však byly lehce v rámci konzultací během vývoje aplikace.

\subsection{Funkční požadavky}
Jsou takové požadavky, které musí být ve výsledné aplikaci implementovány, aby byla splněna požadovaná funkcionalita. Požadavky jsou rozděleny do 8 sekcí označených jako F1 až F7. Jedná se především o funkcionality spojené s nahlašováním poruch nebo požadavků na údržbu spolu s úkony prováděných nad již vzniklými hlášeními. 
\subsubsection{F1: Založení požadavku na údržbu}
\label{sssec:fc_zalozeni_pozadavku}
V případě zjištění potřeby údržby daného vybavení bude uživateli s oprávněním tuto činnost provádět k dispozici založení hlášení v modulu PM s následující editovatelnými parametry.
\begin{itemize}
	\item
	\textbf{Vybavení}: Výběr bude umožněn pomocí hierarchické struktury představující podobu závodu. V případě, že bude uživateli přednastaveno výchozí technické místo (například pracovní linka, hala nebo závod), bude výběrová struktura příslušně omezena. Provedení výběru by mělo být umožněno i pomocí naskenování QR kódu.  
	\item
	\textbf{Příloha}: Ke každému požadavku bude umožněno přiložené jedné přílohy s tím, že prozatím bude omezeno pouze na fotografie (omezený výčet typů souborů). Do budoucna se počítá s rozšířením na vyšší počet příloh.  
	\item
	\textbf{Priorita}: Stanovuje termín, do kterého se požaduje požadovanou údržbu provést. Řešeno pomocí tří úrovní důležitosti, podle kterých se očekává zpracování požadavku v horizontu dne, týdne nebo měsíce.
	\item
	\textbf{Plánovací skupin}: Spočívá ve výběru subjektu zodpovídajícího za údržbu. Fyzicky se jedná o skupinu lidí spravující vymezený okruh údržby (například elektromechanici, mechanici nebo revizní technici). V případě, že bude uživateli přednastavena výchozí plánovací skupina, dojde automaticky k jejímu předvyplnění.
	\item
	\textbf{Pracoviště}: Reprezentuje skupinu údržbářů, kteří jsou podřízeni příslušnému mistru údržby. Z důvodu propojení s modulem CO dochází s návazností na pracoviště k ocenění práce, která je vykazována pomocí zpětného hlášení daným pracovištěm na zakázku PM. 
	\item
	\textbf{Popis poruchy}: Jedná se o stručný popis hlášení, jasně přibližující daný problém (například došlý materiál nebo opotřebení). 
\end{itemize} 
Parametry, které uživatel nebude svévolně volit jsou následující.
\begin{itemize}
	\item
	\textbf{Typ hlášení}: Vychází z prováděné operace, je stanoven konstantou určující typ hlášení požadavku na údržbu.
	\item
	\textbf{Závod}: Definován přihlášeným uživatelem. Každý uživatel bude mít nějaký přidělený, jedná se o potřebný údaj při zakládání hlášení.
\end{itemize} 
Typickým uživatelem využívající tento funkční požadavek bude mistr, vedoucí směny a údržbáři.
\subsubsection{F2: Nahlášení poruchy}
\label{sscec:fc_nahlaseni_poruchy}
Jelikož se technicky na úrovni modulu PM jedná o stejný záznam jako při zakládání požadavku na údržbu, je výčet potřebných parametrů totožný. Nicméně jelikož se hlášení poruchy reálně očekává od jiného typu uživatelů, je výběr následujících parametrů trochu odlišný. Typicky bude poruchu nahlašovat pracovník ve výrobě například na konkrétní lince. 
\begin{itemize}
	\item
	\textbf{Vybavení}: Jelikož se očekává mnohem menší počet vybavení, které bude umožněno uživateli vybrat, nebude se provádět výběr z hierarchické struktury technických míst a vybavení, ale bude k dispozici jenom takové, které spadá pod určité pracoviště. Hierarchické uspořádání však zůstane zachováno. Provedení výběru by mělo být umožněno i pomocí naskenování QR kódu.  
	\item
	\textbf{Plánovací skupin}: Uživatel bude mít na výběr hlášení poruch dvojího typu. Bude na něm, jestli si vybere poruchu s mechanickou nebo elektrotechnickou příčinou. V závislosti na tom bude plánovací skupina přednastavena z uživatelského nastavení.
\end{itemize} 
Needitovatelné parametry budou stejně jako při zakládání požadavku na údržbu přednastaveny z uživatelského nastavení.
\subsubsection{F3: Správa poruch}
\label{sssec:fc_sprava_poruch}
Uživateli musí být k dispozici seznam poruch obsahující potřebné informace (popis hlášení, technické místo spolu s vybavením, pracoviště a informace o tom kdo, kdy poruchu nahlásil a status daného hlášení) pro správné zacházení s nimi. Uživateli budou nad jednotlivými poruchami umožněny následující operace. Některé z nich jsou k dispozici v závislosti na stavu (statusu) hlášení.
\begin{itemize}
	\item
	\textbf{Evidence práce na poruše}: Údržbář (člověk s oprávněním provádět opravy) může na konkrétní poruše zahájit práci, technicky realizována založením PM zakázky, u které díky integraci na modul CO dochází k evidenci nákladů. Takový uživatel může poté práci na svojí zakázce ukončit.
	\item
	\textbf{Zrušení hlášení}: V případě založení poruchy (status odpovídající stavu právě založeno) je umožněno hlášení zrušit. To například pro nevhodné nebo omylné založení. 
	\item
	\textbf{Vyřešení poruchy}: Poté, co se poruše začal někdo věnovat (zahájil na ní práci - došlo k založení zakázky PM) a svou práci ukončil a nikdo jiný už na ní nepracuje, je možné poruchu ukončit. Poté bude údržbář vyzván k odborné specifikaci poruchy. Dostane za úkol specifikovat část objektu, poškození a příčinu. V takovém případ dojde k ukončení celého procesu a dané hlášení již v aplikaci nebude dostupné.
	\item
	\textbf{Výdej náhradního dílu ze skladu}: Pro provedení této činnosti se uživateli přednastaví z uživatelského nastavení závod a sklad s tím, že sklad bude moci změnit. Materiál, množství a možnost poté uživatel zadá ručně. Potvrzením zadaných parametrů dojde k vyskladnění požadovaného materiálu.
	\item
	\textbf{Správa akcí}: Hlášení mají jasný výčet operací, které při práci s poruchou může provést. Jedná se například o informaci objednání náhradního dílu nebo servisu. K takové akci pak může dotyčný uživatel dodat vlastní poznámku. Tyto akce mohou být zakládány, ale i zpětně prohlíženy.
	\item
	\textbf{Zobrazení textů}: Ke každému hlášení je pomocí dlouhých textů v SAPu umožněno přidávat poznámky. V rámci hlášení bude všem dostupná historie těchto poznámek, přidávání bude umožněno v závislosti na typu uživatele.
	\item
	\textbf{Zobrazení přílohy}: V důsledku možnosti přidávat přílohu při zakládání poruchy je i v případě práce s poruchou umožněno si danou přílohu zobrazit a eventuálně stáhnout na lokální disk.
\end{itemize} 
\subsubsection{F4: Správa požadavků na údržbu}
\label{sssec:fc_sprava_poz_udr}
Uživateli musí být k dispozici seznam požadavků obsahující potřebné informace (popis požadavku, technické místo spolu s vybavením, pracoviště, mezní zahájení spolu s ukončením a informace o tom, kdo požadavek založil a status daného hlášení) pro správné zacházení s nimi. Uživateli budou umožněny stejné operace jako pří správě poruch \ref{sssec:fc_sprava_poruch}, kromě změn v následujícím seznamu. Vyřešení poruchy ze správy poruch se požadavků na údržbu netýká.
\begin{itemize}
	\item
	\textbf{Provedení údržby}: Údržbář (člověk s oprávněním provádět opravy) může na konkrétním požadavku zahájit práci, technicky realizovanou založením PM zakázky, u které díky integraci na modul CO dochází k evidenci nákladů. Po ukončení prací na daném požadavku bude moci údržbář rozhodnout, zdali je údržba provedena dostatečně a může dojít předání k operátorům výroby na schválení. 
	\item
	\textbf{Akceptace údržby}: Operátor výroby (člověk s oprávněním provádět opravy) může rozhodnout o dostatečném provedení údržby daného stroje. V takovém případě dojde k ukončení celého procesu a dané hlášení již v aplikace nebude dostupné.
	\item
	\textbf{Reklamování údržby}: Operátor výroby (člověk s oprávněním provádět opravy) může rozhodnout o nedostatečném provedení údržby daného stroje. V takovém případě dojde k navrácení hlášení údržbářům, aby mohli na dané údržbě znovu pracovat.
\end{itemize} 
\subsubsection{F5: Správa prevencí}
\label{sssec:fc_sprava_prev}
Uživateli musí být k dispozici seznam prevencí obsahující potřebné informace (popis prevence, technické místo spolu s vybavením, pracoviště, mezní ukončením a informace o statusu daného hlášení) pro správné zacházení s nimi. Uživateli budou umožněny stejné operace jako pří správě požadavků na údržbu \ref{sssec:fc_sprava_poz_udr}, kromě změn v následujícím seznamu. 
\subsubsection{F6: Zobrazení dokumentace ke stroji (vybavení)}
\label{sssec:fc_zobrazeni_dokumentace}
V závislosti na vybraném technickém místě nebo vybavení dojde k zobrazení seznamu přiložené dokumentace. Ta je uložena na sdíleném uložiti v interní síti společnosti a bude tedy dostupná pouze v případě použití aplikace uvnitř dané síti. Výběr technického místo nebo vybavení bude umožněn z hierarchické struktury.
\subsubsection{F7: Administrace uživatele}
\label{sssec:fc_administrace}
Bude umožněna obecná správa účtu uživatelů, tedy základní operace jako přidání nebo odebrání účtu, nastavení jména uživatele a osobního čísla odpovídajícímu osobnímu číslu (identifikátor zaměstnance) v ERP. Administrátorský účet bude moci měnit nastavení ostatních účtů, bude přiřazovat uživatelské role (oprávnění k zacházení s jednotlivými funkcionalitami) a parametry charakterizující daného uživatele. Pod tím se schovává nastavení závodu, plánovací skupiny, předdefinovaného technického místa a dalších atributů ulehčujících uživateli práci s aplikací (například přednastavené hodnoty pro výběr pracovišť, plánovacích skupin při zakládání požadavků na údržbu nebo hlášení poruch). V případě ztráty uživatelova hesla, bude z administrátorského účtu umožněno inicializování hesla.

\subsection{Nefunkční požadavky}
Jsou takové požadavky, které nejsou přímo nutné pro splnění požadované funkcionality, nicméně vhodné pro správný chod aplikace. Jedná se například o specifikaci očekávání od designu, zabezpečení nebo dostupnosti systému a dalších pasivních požadavků. Navržené požadavky pro výslednou aplikaci jsou rozděleny do 5 sekcí označených jako N1 až N4.
\subsubsection{N1: Grafické uživatelské rozhraní}
Uživatelské rozhraní bude dostupné v českém jazyce. Nicméně pro budoucí plánované využití i v zahraničních závodech se očekává snadné rozšíření do ostatních jazyků jako je například angličtina nebo němčina. Jelikož se nejedná o první organizační aplikaci pracující nad nějakým z modulů SAPu, požaduje se zachování stejného UI frameworku SAPUI5.
\subsubsection{N2: Dostupnost}
Aplikace musí být viditelná v internetu, musí být tedy dostupná z veřejné internetové adresy. Aplikace musí být ovšem plně funkční i ve vnitřní síti, která nemá přístup do internetu. Veškeré zdroje aplikace musí bý tedy uloženy na interním serveru poskytujícího run-time prostředí pro webovou aplikaci.
\subsubsection{N3: Spolehlivost a spravovovanost}
Aplikace bude umožňovat logování činnosti uživatelů v systému z důvodu lepší identifikaci chyb. A to minimálně z počátku provozu aplikace. Taktéž bude umožněno I v případě vyššího zatížení musejí výt veškeré transakční kroky prováděné uživatelem uskutečněny.
\subsubsection{N4: Bezpečnost}
Z bezpečnostních důvodů musí v aplikaci docházet k autentizaci a autorizaci každého uživatele. Uživatel bude v aplikaci smět dělat pouze ty úkony, které mu administrátor povolí. Komunikace napříč komponentami musí být šifrována.

\section{Model případů užití (Use Case Model)}
Jedná se o detailní specifikaci funkčních požadavků. Typicky slouží pro tvorbu uživatelské příručky, jako podklady k tvorbě akceptačních testů, zpřesnění odhadů pracnosti a zadání pro programátora. Zahrnuje informace o tom, kdo bude se systémem pracovat a jaké funkcionality využívat. K tomu slouží vydefinovaný seznam účastníků a diagramy případů užití.

\subsection{Seznam účastníků}
\label{ssec:seznam_ucastniku}
Níže zmínění účastníci odpovídají standardnímu pracovnímu modelu stanovenému v organizaci.
\begin{itemize}
	\item
	\textbf{Operátor výroby}: Vychází z navržené uživatelské role operátor výroby \ref{ssec:operator_vyroby}. Na přiděleném pracovišti provádí přidělenou výrobní činnost. 
	\item
	\textbf{Údržbář}: Odpovídá navržené roli údržbář \ref{ssec:udrzbar}, zpravidla bude disponovat i rolí pro zakládání požadavků na údržbu \ref{ssec:pozadavek_udrzba}. 
	\item
	\textbf{Správce / Administrátor}: Vychází z navržené uživatelské správce / administrátor \ref{ssec:administrator}.
\end{itemize} 	
Nicméně nic administrátorovi nebrání tomu role různě kombinovat, nemají mezi sebou výlučný vztah. Tudíž administrátor může mít klidně i role údržbáře a operátora výroby, čímž je schopen provádět jejich příslušné operace.

\subsection{Diagram případů užití}
Souží pro detailní specifikaci požadavků na systém s tím, že graficky zobrazuje účastníky a jejich příslušná oprávnění. V následujících podkapitolách jsou vytvořeny diagramy pro nejdůležitější procesy očekávané d výsledné aplikace.
\label{ssec:diagram_pripadu_uziti}
\subsubsection{UC1: Správa poruch}
\label{sssec:uc_sprava_poruch}
Následující případ užití týkající se správy poruch zahrnuje funkční požadavky pro hlášení poruch \ref{sscec:fc_nahlaseni_poruchy} a jejich následnou správu \ref{sssec:fc_sprava_poruch}.
\begin{figure}[H]
	\centering
	\includegraphics[width=1\textwidth]{images/ea_sprava_poruch.jpg}
	\caption{Diagram případu užití pro správu poruch}
	\label{img:uc_sprava_poruch}
\end{figure}
\subsubsection{UC2: Správa požadavků na údržbu}
\label{sssec:uc_sprava_udrzby}
Následující případ užití týkající se správy poruch zahrnuje funkční požadavky pro hlášení poruch \ref{sscec:fc_nahlaseni_poruchy} a jejich následnou správu \ref{sssec:fc_sprava_poruch}.
\begin{figure}[H]
	\centering
	\includegraphics[width=1\textwidth]{images/ea_sprava_poruch.jpg}
	\caption{Diagram případu užití pro správu poruch}
	\label{img:uc_sprava_poruch}
\end{figure}
\subsubsection{UC3: Správa prevencí}
\label{sssec:uc_sprava_prevenci}
Následující případ užití týkající se správy poruch zahrnuje funkční požadavky pro hlášení poruch \ref{sscec:fc_nahlaseni_poruchy} a jejich následnou správu \ref{sssec:fc_sprava_poruch}.
\begin{figure}[H]
	\centering
	\includegraphics[width=1\textwidth]{images/ea_sprava_poruch.jpg}
	\caption{Diagram případu užití pro správu poruch}
	\label{img:uc_sprava_poruch}
\end{figure}

\chapter{Návrh uživatelského rozhraní}
Tato kapitola se věnuje průběhu vytváření návrhu uživatelského rozhraní výsledné aplikace. Na základě funkčních požadavků vzešlých z prvních schůzek se zadavatelem a modelu případu užití jsou jednotlivé vzniklé úkony sdruženy v task group. Na základě task group je vytvořen task graph, který v grafické podob přiřazuje funkcionality k navrženým obrazovkám. Na základě toho je zde vytvořen Lo-Fi prototyp. K tomu byly použity dva nástroje Built.me a Balsamiq Mockups, které jsou následně porovnány. Na základě výsledků z testování Lo-Fi prototypu s uživateli jsou odpovídající změny zaneseny do výsledné aplikace.

\section{Task Groups}

\section{Task Graph}
\begin{figure}[H]
	\centering
	\includegraphics[width=1\textwidth]{images/task_graph.png}
	\caption{Diagram případu užití pro správu poruch}
	\label{img:uc_sprava_poruch}
\end{figure}


\section{Lo-Fi prototyp}

\subsection{Balsamiq}

\subsubsection{Údržbář}

\paragraph{Desktopová verze}

\begin{figure}[H]
	\centering
	\includegraphics[width=1\textwidth]{images/bal_login}
	\caption{Diagram případu užití pro správu poruch}
	\label{img:uc_sprava_poruch}
\end{figure}

\begin{figure}[H]
	\centering
	\includegraphics[width=1\textwidth]{images/fiori_launchpad}
	\caption{Diagram případu užití pro správu poruch}
	\label{img:uc_sprava_poruch}
\end{figure}

\begin{figure}[H]
	\centering
	\includegraphics[width=1\textwidth]{images/bal_homepage}
	\caption{Diagram případu užití pro správu poruch}
	\label{img:uc_sprava_poruch}
\end{figure}


\begin{figure}[H]
	\centering
	\includegraphics[width=1\textwidth]{images/bal_poruchy_seznam}
	\caption{Diagram případu užití pro správu poruch}
	\label{img:uc_sprava_poruch}
\end{figure}

\begin{figure}[H]
	\centering
	\includegraphics[width=1\textwidth]{images/bal_poruchy_seznam_zalozeni_poruchy}
	\caption{Diagram případu užití pro správu poruch}
	\label{img:uc_sprava_poruch}
\end{figure}

\paragraph{Mobilní verze}

\begin{figure}[H]
	\centering
	\includegraphics[width=1\textwidth]{images/bal_login_hompage_mob}
	\caption{Diagram případu užití pro správu poruch}
	\label{img:uc_sprava_poruch}
\end{figure}

\begin{figure}[H]
	\centering
	\includegraphics[width=1\textwidth]{images/bal_poruchy_mob}
	\caption{Diagram případu užití pro správu poruch}
	\label{img:uc_sprava_poruch}
\end{figure}

\subsection{Built}

\paragraph{Desktopová verze}

\begin{figure}[H]
	\centering
	\includegraphics[width=1\textwidth]{images/bu_poruchy_seznam}
	\caption{Diagram případu užití pro správu poruch}
	\label{img:uc_sprava_poruch}
\end{figure}

\begin{figure}[H]
	\centering
	\includegraphics[width=1\textwidth]{images/bu_zalozeni_poruchy}
	\caption{Diagram případu užití pro správu poruch}
	\label{img:uc_sprava_poruch}
\end{figure}

\paragraph{Mobilní verze}

\begin{figure}[H]
	\centering
	\includegraphics[]{images/bu_poruchy_seznam_mob}
	\caption{Diagram případu užití pro správu poruch}
	\label{img:uc_sprava_poruch}
\end{figure}

\begin{figure}[H]
	\centering
	\includegraphics[]{images/bu_zalozeni_poruchy_mob}
	\caption{Diagram případu užití pro správu poruch}
	\label{img:uc_sprava_poruch}
\end{figure}

\subsection{Heuristická analýza}
Při návrhu uživatelského rozhraní je dobré držet se deseti následujících pravidel z Nielsenovi heuristické analýzy. V této kapitole je čerpáno ze zdrojů [8] a [9]. Nielsenova heuristická analýza je jednou ze základních metod pro testování uživatelského rozhraní. Jedná se o seznam pravidel, které by mělo uživatelské rozhraní splňovat. Jakob Nielsen a Rolf Molich v roce 1990 vytvořili heuristiku
pro heuristické vyhodnocení a poté v roce 1994 Jakob Nielsen revidoval tuto heristiku na množinu pravidel.

\begin{enumerate}
	\item
	\textbf{Viditelnost stavu systému} - Uživatel by měl být vždy systémem vhodně informován (v rozumném čase)
	o tom co se zrovna děje. Systém by tak měl reagovat na uživatelský vstup, nebo v případě, že se například provádí nějaký časově náročnější výpočet nebo
	stahování dat, tak zobrazit progress bar.
	\item
	\textbf{Propojení systému a reálného světa} - Systém by měl na uživatele mluvit jazykem uživatele se slovy, frázemi a koncepty, které jsou uživateli známé, zná je z reálného světa. Neměly by se využívat
	pojmy, které jsou například specifické pouze pro daný systém.
	\item
	\textbf{Uživatelská kontrola a svoboda} - Uživatelé se často učí nové funkce systému pomocí chyb, které provedou. Když uživatelé udělají chybu, musí mít možnost provedenou akci vrátit zpět a vrátit
	tak systém do předchozího stavu. V případě, že se provádí nenávratná akce,
	je třeba na to uživatele řádně upozornit.
	\item
	\textbf{Standardizace a konzistence} - Uživatel nesmí být zmaten různými termíny v různých situacích, přestože mají dané termíny stejný význam. Systém by měl vždy dodržovat standardy, které
	jsou na dané platformě. Proto je například potřeba dodržovat standardní chybové
	hlášky, správné umístění komponent apod. Ideální je používat standardní
	platformové komponenty.
	\item
	\textbf{Prevence chyb} - Lepší než dobré chybové hlášky je návrh, který zabraňuje samotnému výskytu
	chyb. Buď je možné podmínky náchylné k chybám co nejvíce eliminovat, nebo
	na chyby uživatele upozornit ještě dříve než například potvrdí formulář.
	\item
	\textbf{Rozpoznání namísto vzpomínání} - Je třeba minimalizovat zatěžování paměti uživatele tím, že uživatel vždy vidí potřebné informace a akce, které může provést. Uživatel si tak například nemusí
	pamatovat informace z jedné části formuláře na další.
	\item
	\textbf{Flexibilní a efektivní použití} - Systém by měl v závislosti na jeho možnostech a typu umožňovat dvě verze ovládání. Pro méně zkušené uživatele a pro zkušené uživatele. Verze pro méně zkušené uživatele by měla obsahovat pouze základní funkce a možnosti nastavení tak, aby „nezkušený“ uživatel nebyl zbytečně zatěžován funkcionalitami, které stejně nepotřebuje. Naopak pro zkušené uživatele by se	měli zobrazit všechny funkcionality, včetně těch složitějších. Zkušenější uživatel by měl mít také případně možnost si potřebné funkcionality přizpůsobit pomocí maker. Pro oba typy uživatelů je také dobré umožnit využívat klávesové zkratky
	\item
	\textbf{Estetický a minimalistický} - Uživatel by měl mít co nejméně možností kam může kliknout, protože každá
	další možnost soutěží o pozornost uživatele. Čím méně možností uživatel má,
	tím rychleji je schopen pokračovat. Na obrazovce by také měly být zobrazeny
	pouze informace, které uživatel v dané situaci opravdu potřebuje.
	\item
	\textbf{Pomoc uživatelů pochopit, poznat a vzpamatovat se z chyb} - Chybové hlášky by měly být v přirozeném jazyce a neměly by například obsahovat
	žádné chybové kódy apod. Hlášky by měly přesně popisovat co je za
	problém a doporučit uživateli jak pokračovat dál. Ideální je, když uživatel
	nemá možnost dojít do stavu, kdy je potřeba chybové hlášení.
	\item
	\textbf{Nápověda a návody} - Přestože je lepší, když je systém použitelný bez jakékoliv nápovědy, nápovědu
	by měl systém obsahovat. Veškeré informace by v systému měly být snadno
	vyhledatelné a obsahem nápovědy by měly být názorné příklady.
\end{enumerate} 

\section{Porovnaní prototypovacích nástrojů}


\chapter{Implementace}

\begin{figure}[H]
	\centering
	\includegraphics[width=1\textwidth]{images/architektura}
	\caption{Diagram případu užití pro správu poruch}
	\label{img:uc_sprava_poruch}
\end{figure}

Tato kapitola se věnuje implementaci jednotlivých částí navržené architektury \ref{sec:architektura}, která je ihned v první podkapitole zobrazena a popsána. Podrobnější popis jednotlivých komponent následuji ihned poté. Architekrura je rozdělena do tří bloků a tomu odpovídají i tři následující podkapitoly. 

\section{SAP ERP}
Touto části architektury se tato práce přímo nezabývá, nicméně je zde pro lepší představu o celkovém systému krátce popsána. Jak již bylo v kapitole věnující se teoretickému základu řečeno, základem veškerých potřebných procesů je SAP ERP. Konkrétně potom moduly PM, MM a CO. Každý z těchto modulů ma desítky BAPI (Business Application Programming Interface) funkcí představujících rozhraní pro základní operace nad daným modulem. Správné sekvence a data volaných BAPI funkcí jsou potom obaleny do funkčních modulů, které umožňují vzdálené vyvolání z jiných systémů. Ke vzdálenému volání slouží v rámci SAP systému RFC (Remote Function Call), které umožňuje přenášet data napříč jednotlivými systémy. Takovýchto funkcí vzniklo v ERP systému více než 20, výčet modulů je k vidění na obrázku \ref{img:erp_fm} níže.

\begin{figure}[H]
	\centering
	\includegraphics[width=1\textwidth]{images/erp_fm}
	\caption{Seznam funkčních modulů v ERP systému}
	\label{img:erp_fm}
\end{figure}

\section{SAP GW}
Jak již bylo zmíněno v kapitole věnující se tomuto produktu více teoreticky, primárním účelem serveru je komunikace s okolním světem. Standardní cesta pro Fiori aplikace je používáním servisů ODATA. Ne všichni zákazníci ovšem mají nakoupené licence pro provoz Fiori touto cestou a právě proto byla navržena architektura založená na technologii BSP, běžně dostupné v produktech GW, ale i ERP. Na obrázku \ref{img:gw_json} je seznam 36 BSP stránek, které pro aplikaci vznikly. V porovnání s počtem funkčních modulů na straně ERP je evidetní, že nejsou vytvořeny v poměru 1:1. Je to především z toho důvodu, že jsou v této úrovni uloženy uživatelská data portálových uživatelů. 

\begin{figure}[H]
	\centering
	\includegraphics[width=1\textwidth]{images/gw_json}
	\caption{Seznam funkčních modulů v ERP systému}
	\label{img:gw_json}
\end{figure}

Všechny výše zobrazené BSP stránky jsou implementovány na stejném principu. Využívají události OnRequest, ve které dojde ke zpracování vstupních dat a vygenerování adekvátních dat na výstup. V první fázi jsou vytvořeny lokální proměnné potřebné pro správné zpracování dat. Následně je deserializován vstupní objekt typu JSON do odpovídajících struktur programovacího jazyka ABAP. Poté je pomocí RFC zavolán příslušný funkční modul. Stejně tak jako v ukázce algoritmu \ref{code:rfc_call} níže. 

\begin{algorithm}[H]	
	\begin{lstlisting}[language = VHDL]  
CALL FUNCTION 'ZITL_PM_CREATE_NOTIF' DESTINATION lv_dest
  EXPORTING
    is_notif_get = zitl_input_json-qmart
    it_tplnr     = zitl_input_json-tplnr
  IMPORTING
    et_notif     = lt_notif
    ev_error     = lv_error.
	\end{lstlisting}
	\caption{Vytvoření instance třídy PortList}	
	\label{code:rfc_call}
	\small Kód zobrazuje volání funkčního modulu ZITL\_MOB\_PM\_GET\_NOTIF pro načtení seznamu hlášení. Parametrem DESTINATION je řízeno směrování volání. Hodnotou musí být nastavené spojení se vzdáleným servem zabezpečeného pomocí metody BASIC. Parametry EXPORTING a IMPORTING určují vstupní a výstupní proměnné z funkčního modulu. Vstupními parametry je tak řízen například typ hlášení, který určuje zdali se jedná o poruchu, prevenci nebo požadovanou údržbu. Obsahuje však i další parametry určují datumový rozsah a podobně.
\end{algorithm}	

Načtená data jsou zpětně serializována pro výstup. Obsahem celé BSP HTML stránky je tedy objekt typu JSON. Jak již ale bylo zmíněno, na úrovni GW se nacházejí i uživatelská data představující portálové uživatele. Na následujícím obrázku \ref{img:gw_db} je k vidění ilustrační databázové schéma. 

\begin{figure}[H]
	\centering
	\includegraphics[]{images/gw_db}
	\caption{Ilustrační databázové schéma pro uživatelská data portálového uživatele}
	\label{img:gw_db}
\end{figure}

Databázové schéma neodpovídá úplně přesně skutečnosti. Některá pole a vazby tabulek jsou úmyslně skryta a to buď protože nejsou pro celý koncept aplikace relevantní nebo z důvodu zvýšení bezpečnosti. Ze schematu vychází vazby 1 ku N pro všechny kombinace kmenové tabulky PMP\_USER a pomocných tabulek PMP\_USER\_ROLES, PMP\_USER\_ATTRUBUTES a PMP\_USER\_LOCATIONS. Trochu podrobněji jsou jednotlivé tabulky popsány v následujícím výčtu.

\begin{itemize}
	\item
	\textbf{PMP\_USER}: Tabulka obsahuje záznamy jednotlivých portálových uživatelů. Jedná se tedy především o přihlašovací údaje a jednotlivé informace přímo související s uživatelem. Jako je například jméno, pod kterým uživatel ve webové aplikaci vystupuje, osobní číslo v ERP, informace o tom zdali se jedná o správce nebo také platnosti uživatele a data s časy posledních změn spolu s jejich autory.
	\item
	\textbf{PMP\_USER\_ROLES}: Tabulka s cizím klíčem uživatel. Primární klíč je identifikátor role, která je uložena ve standartních SAP tabulkách, které jsou zpracovatelné pomocí standardních transakcí.
	\item
	\textbf{PMP\_USER\_ATTRIBUTES}: Tabulka s cizím klíčem uživatele. Primární klíč je identifikátor atributu. Výčet atributů není nikterak stanoven a zodpovědnost za správné vyplnění je přenesena na správce účtů, který má k dispozici seznam užitečných atributů pro webovou aplikaci. Součástí tabulky je i pole pro hodnotu takového atributu. S vytvořením aplikační logiky kontrolující správnost atributů z pevně stanového výčtu je počítáno v budoucím rozšíření aplikace.
	\item
	\textbf{PMP\_USER\_LOCATIONS}: Tabulka s cizím klíčem uživatele. Primární klíč je identifikátor zodpovědného pracoviště. Ten není prozatím nikterak kontrolován vůči hierarchickému uspořádání technických míst v modulu SAP PM. S vytvořením aplikační logiky kontrolující správnost pracovišť ze stanovené hierarchie je počítáno v budoucím rozšíření aplikace.
\end{itemize} 

\section{Apache Tomcat}
Apache Tomcat je známý open-source webový server a servletový kontejner. Jedná se o oficiální referenční implementaci technologií Java Servlet a Java Server Pages (JSP). Na serveru mohou běžet uživatelské servlety (programy napsané v Javě), které umí zpracovávat požadavky zasílané pomocí HTTP protokolu a tímtéž protokolem na ně odpovídat. Apache Tomcat zde slouží jako zásobník servletů starající se o jejich spouštění, běh, ukončování a podobně.

\subsection{Login Modul}
Login modul využívá systému \textbf{JAAS} (Java Authentication and Authorization Service), který slouží pro autentizaci a autorizaci uživatele. Jak již z názvu vyplývá, jedná se o bezpečnostní systém založený na technologii Java. Vyskytuje se od verze 1.4 v J2EE (Java 2 Enterprise Edition). 

Deklarativní zabezpečení J2EE chrání webové aplikace podle aktuálního vzorce uživatelovi URL. Cesta k souborům může být popsána absolutním i relativním způsobem. Jeli například požadováno povolení přístupu k aplikaci uživatelům s rolí USERS, je zapotřebí v definici web.xml \ref{item:pm_web_inf} zavést následující definici.

\begin{algorithm}[H]	
	\begin{lstlisting}[language = XML]  
<security-constraint> 
  <web-resource-collection> 
    <web-resource-name>AllPublic</web-resource-name> 
    <url-pattern>/</url-pattern> 
  </web-resource-collection> 
  <auth-constraint> 
    <role-name>USERS</role-name> 
  </auth-constraint> 
</security-constraint>
	\end{lstlisting}
	\caption{Definice zabezpečení aplikace pro roli USERS}	
	\label{code:j2ee_definition}
	\small Tato definice chrání webovou aplikaci od kořenového adresáře J2EE, jak je naznačeno vzorem \uv{\textbackslash}. Pokud by mělo být zabezpečení aplikováno jen na část aplikace, je třeba uvést za \uv{\textbackslash} jméno adekvátního podadresáře.
\end{algorithm}	

Ověřování v deklarativní ochraně je vynuceno, právě tehdy když si uživatel vyžádá chráněnou oblast webové aplikac. Pokud nebyl dříve ověřen, zobrazí se přihlašovací dialog, aby se uživatel mohl identifikovat. Běžně používané způsoby ověření jsou FORM a BASIC. 

\paragraph{BASIC} Je základním typem autentizace. Využívá standardního dialogu prohlížeče pro vložené uživatelského jména a hesla. Tento dialog nelze nikterak modifikovat, proto se jeho vzhled liší pouze na typu aktuálně používaného prohlížeče. Uživatelská pověření pro autentizovanou oblast jsou uložena v rámci session prohlížeče. Uživatelská data (jméno a heslo) jsou potom v zakodované formě posílána s každým HTTP requestem.

\begin{algorithm}[H]	
	\begin{lstlisting}[language = XML]  
<login-config> 
  <auth-method>BASIC</auth-method> 
</login-config>
	\end{lstlisting}
	\caption{Definice typu autentizace BASIC}	
	\label{code:auth_basic_def}
\end{algorithm}	

\paragraph{FORM} Je přizpůsobivější variantou autentizace. Umožňuje vývojáři specifikovat vlastní dialog pro přihlášení. Jediné omezení spočívá v pojmenování vstupní hodnoty uživatelského jména na j\_username a hesla na j\_password. Přihlašovací akce pro autentizaci potom musí mít hodnotu j\_security\_check v rámci J2EE kontejneru. Uživatel posléze zůstává ověřen přes session server.

\begin{algorithm}[H]	
	\begin{lstlisting}[language = XML]  
<login-config> 
  <auth-method> FORM</auth-method> 
  <form-login-config> 
    <form-login-page>/login.html</form-login-page>
    <form-error-page>/login_error.html</form-error-page>
  </form-login-config> 
</login-config>
	\end{lstlisting}
	\caption{Definice typu autentizace FORM}	
	\label{code:auth_form_def}
	\small Navíc od autentizace typu BASIC jsou zde zadefinovány dvě html stránky. Element <form-login-page> stanovuje stránku, která je uživateli zobrazena při prvotním pokusu uživatele o načtení aplikace, stránka stanovená elementem <form-error-page> se zobrazí v případě, že uživatel zadá špatnou kombinace jména a hesla nebo se v průběhu autentizace vyskytne jiná chyba.
\end{algorithm}	

Oba přístupy autentizace jsou zranitelná vůči útokům skrze odposlouchávání. Proto je silně doporučeno zpřístupňovat aplikaci skrze HTTPS protokol. 

Kompletní proces autentizace a autorizace je implementován pomocí tříd napsaných v programovacím jazyce Java. Výčet tříd odpovídá struktuře \ref{img:loginmodule_structure} zobrazené níže. 

\begin{figure}[H]
	\centering
	\includegraphics[]{images/loginmodule_structure}
	\caption{Struktura paketu realizující autentizaci a autorizaci}
	\label{img:loginmodule_structure}
\end{figure}

Nejdůležitější třídou je \textbf{PMLoginModule}, která spočívá v přepisu čtyř procesních metod. Inicializační metodou je \textbf{initialize}, která nastavení instančním proměnným výchozí hodnoty. Prvním z nich je základní objekt CallbackHandler, sloužící k předávání hodnot mezi modulem a uživatelovým prohlížečem. Druhým je objekt Subject udržující informace o uživateli. Takovou hodnotou je například identifikátor session. Metodou realizující autentizaci je metoda \texttt{\textbf{Login}}, která pomocí objektu CallbackHandler získá uživatelem vyplněné hodnoty jména a hesla, jejíž ověření se právě v rámci této metody musí provést. Pro kontrolu očekávaného slouží statická třída \textbf{Password}, která obsahuje metody pro generování hesel i ověření jejich shodnosti. Výstupem je potom boolean hodnota true nebo false. O autorizaci se stará metoda \texttt{\textbf{Commit}}, jejíž úkolem je přidělení příslušných rolí uživateli. K tomu slouží třídy \texttt{\textbf{UserPrincipal}} a \textbf{RolePrincipal} reprezentující jednotlivé role a uživatele. Poslední metodou k doplnění celého procesu je \textbf{Logout}, která má za úkol odstranit uživateli role a a následně celý obejt Subjekt, který ho reprezentuje.

\paragraph{Nasazení} Celý projekt je zapotřebí zabalit do archivu typu JAR, sloužící pro distribuci programů a knihoven. Následně ho vložit do run-time prostředí Tomcat serveru. Pro uložení takové knihovny je určen adresář libs.

K tomu, aby server vůbec věděl, že má pro aplikace vyžadující autentizaci uživatele použít vytvořenou JAR knihovnu, je zapotřebí definovat JAAS konfigurační soubor obsahující cestu k dané knihovně. Ukázka takové definici je v kódu \ref{code:jaas_config} níže. 

\begin{algorithm}[H]	
	\begin{lstlisting}[language = VHDL]  
pmloginmodule {
  ite.cz.mo.pmlogin.PMLoginModule required debug=true;
};
	\end{lstlisting}
	\caption{Konfigurační soubor JAAS}	
	\label{code:jaas_config}
\end{algorithm}	

Cestu souboru je pak zapotřebí definovat v Java parametrech serveru. K určení konfiguračního souboru slouží parametr Djava.security.auth.login.config. Hodnotou k tomuto parametru je absolutní cesta požadovaného souboru.

\subsection{GW Servlet}
Na obrázku \ref{img:gw_servlet_structure} je k vidění struktura paketu realizujícího mezivrstvu pro komunikace s SAP GW. 
\begin{figure}[H]
	\centering
	\includegraphics[]{images/gw_servlet_structure}
	\caption{Struktura paketu realizujícího middleware vrstvu mezi PM SAPUI5 aplikací a SAP GW}
	\label{img:gw_servlet_structure}
\end{figure}
Jedná se o čtyři třídy napsané v programovacím jazyce Java. Podrobněji jsou jednotlivé třídy popsány v seznamu níže.

\begin{itemize}
	\item
	\textbf{GWServlet}: Je Java třída implementující interface javax.servlet.Servlet sloužící pro zpracování Http požadavků. Základem tohoto servletu je přepsaná metoda doGet, která obsahuje parametry HttpServletRequest a HttpServletResponse umožňují jednak načtení přijatých dat společně s dalšími parametry, které s sebou požadavek nese a potom také adekvátní data pomocí odpovědi vrátit. Metoda doGet je volána z další přepsané metody doPost, která je vyvolána v případě HTTP requestu typu POST. 
	
	\begin{algorithm}[H]	
		\begin{lstlisting}[language = VHDL]  
CALL FUNCTION 'ZITL_MOB_PM_CREATE_NOTIF' DESTINATION lv_dest
EXPORTING
is_notif_get = zitl_input_json-qmart
it_tplnr     = zitl_input_json-tplnr
IMPORTING
et_notif     = lt_notif
ev_error     = lv_error.
		\end{lstlisting}
		\caption{Vytvoření instance třídy PortList}	
		\label{code:rfc_call}
		\small Kód zobrazuje volání funkčního modulu ZITL\_MOB\_PM\_GET\_NOTIF pro načtení seznamu hlášení. Parametrem DESTINATION je řízeno směrování volání. Hodnotou musí být nastavené spojení se vzdáleným servem zabezpečeného pomocí metody BASIC. Parametry EXPORTING a IMPORTING určují vstupní a výstupní proměnné z funkčního modulu. Vstupními parametry je tak řízen například typ hlášení, který určuje zdali se jedná o poruchu, prevenci nebo požadovanou údržbu. Obsahuje však i další parametry určují datumový rozsah a podobně.
	\end{algorithm}	
	
	
	\item
	\textbf{Node}: Třída reprezentující prvek hierarchie technických míst a vybavení z modulu SAP PM. Jelikož se v interně technické místo a vybavení datově liší, je vytvořena tato třída, která nesrovnalosti mezi těmito objekty eliminuje. Datově rozdílné identifikátory jsou nahrazeny jednotným id a ostatní lišící se parametry jsou sloučeny do atributů třídy odpovídajícím významu původního datového objektu. To posléze umožňuje v prezentační vrstvě lehce pracovat v 
	\item
	\textbf{TplnrHierarchy}:
	\item
	\textbf{Password}:
\end{itemize} 

\section{PM SAPUI5 Aplikace}
V následujících podkapitolách je popsána implementační struktura projektu pro PM SAPUI5 Aplikaci. Jelikož se jedná o stěžejní část celkové architektury, obsahuje veškerou logiku očekávanou od prezentační vrstvy. Jsou zde zadefinovány veškeré pohledy (stránky), se kterými se uživatel bude moci setkat.  

\subsection{Struktura aplikace}
Zde zobrazena a popsána celková struktura aplikace z vývojářského pohledu. Na obrázku \ref{img:pmfiori_app_struct} je k vidění hlavní strom adresářů a souborů v použitém Eclipse projektu. 

\begin{figure}[H]
	\centering
	\includegraphics[]{images/fiori_app_struct}
	\caption{Struktura PM aplikace s hlavní prezentační logikou}
	\label{img:pmfiori_app_struct}
\end{figure}

Jednotlivé položky jsou potom popsány v následujícím seznamu. Koncepčně je popis strukturován tak, že nejdříve jsou napsány obecné vlastnosti očekávané od daného adresáře nebo souboru a poté jsou dopsány případné poznámky z implementace.

\begin{itemize}
	\item
	\textbf{WEB-INF}: \label{item:pm_web_inf} Adresář obsahuje externí knihovny potřebné pro správnou funkčnost aplikace. Jelikož základní datovou reprezentací používanou v této aplikaci je JSON a programovací jazyk Java nativně práci s tímto formátem nepodporuje, je v tomto adresáři přiložena externí knihovna \textbf{java-json.jar}, která práci s tímto typem objektů umožňuje. Dále je zde uložen soubor \textbf{web.xml} popisující nasazení webové aplikace včetně jejich webových služeb. Slouží k deklaraci servletů a filtrů používaných danou webovou službou. V případě této aplikace se jedná například o GWServlet.
	\item
	\textbf{view}: Složka obsahuje zadefinovaná view (UI rozvržení elementů) ve značkovacím jazyce XML. Hlavním úkolem těchto souborů je hierarchické rozvržení použitých komponent jednotlivých stránek. Pomocí různých layoutů tak umožňuje jasně definovat, že dané view reprezentuje dialog, jehož obsahem je formulář o pěti prvcích, z nichž každý používá jinou vstupní uživatelskou komponentu. Podrobněji se jednotlivým souborům věnuje podkapitola views. 
	\item
	\textbf{controller}: Adresář obsahující controllery napsané v programovacím jazyce JavaScript. Slouží především pro obsloužení uživatelské interakce s aplikací. Klikne-li například uživatel na tlačítko přidat uživatele, právě zde se nachází část kódu, která požadovanou funkcionalitu provede. Dojde například k přípravě dat pro dialog obsahující potřebná pole pro založení nového uživatele. Jednotlivým souborům se podrobněji věnuje kapitola controllery níže.
	\item
	\textbf{css}: Obsahuje soubory pro úpravu kaskádových stylů. Jelikož aplikace využívá z drtivé většiny pouze předdefinované styly frameworku SAPUI5, je zde pouze jeden soubor style.css obsahující pár úprav oproti standardu.
	\item
	\textbf{i18n}: Pro internacionalizaci se používá numeronymum i18n a v tomto případě adresář obsahuje soubory s dvojicemi hodnot klíč - hodnota, které slouží pro zobrazení v textu požadovaného jazyku. K rozlišení jazyků se používá postfix v názvu souborů. Pro český jazyk je to například název i18n\_cs.properties, určený dle koncovky \_cs. Seznam koncovek odpovídá \textbf{standardu ISO 639-2}.	
	\item
	\textbf{model}: Obsahuje pomocné JavaScriptové soubory s funkcemi, které se opakovaně používají napříč celou aplikací. Jedná se například o formátovací funkce, stanovení modelů obsahující informace o používaném zařízení uživatele a podobně. V mém případě obsahuje tři následující soubory.
	\begin{itemize}
		\item
		\textbf{models.js}: Slouží pouze k zadefinování modelu s informace o používaném zařízení. Poskytuje tak napříč zbytku aplikace například aktuální šířku displeje, používaný operační systém nebo typ prohlížeče. Na základě toho poté dochází ve zbytku aplikace k používání komponent příslušných dané velikosti displeje, nedochází tak k zobrazování tabulky na mobilním zařízení, ale k adekvátně upravenému listu záznamů.
		\item
		\textbf{formatter.js}: Obsahuje funkce pro úpravu zobrazované informace získaných z backendu. SAPí interní formát data 2018-05-04, lze tak pomocí takových funkcí převést na datum v požadovaném tvaru a naopak.
		\item
		\textbf{utils.js}: \label{file:utils.js} Disponuje především funkcemi pro určení, zdali se má daná komponenta zobrazovat. Například tlačítko pro schválení údržby u operátora výroby musí být zobrazené pouze v případě, že na něm údržbáři dokončili práce. Na základě statusu hlášení poté funkce vrací boolean hodnoty true nebo false pro zobrazení daného tlačítka. Dále jsou zde funkce pro přeposílání dat na GWServlet využívající AJAX, který umožňuje asynchronní komunikaci pro výměnu dat s backendem. 
		\begin{algorithm}[H]
			\begin{lstlisting}[language=java]      
query : function(data, success, error) {
  $.ajax({
    type : 'POST',
    url : url,
    cache : false,
    async : true,
    data : data,
    dataTye : "json",
    success : success,
    error : error
  });
},
			\end{lstlisting}
			\caption{Ukázka AJAX volání}	
			\label{code:ajax}
			\small Tato funkce se používá u každého volání z aplikace na GWServlet pro následné zpracování na backendu. Na vstupu jsou tři parametry. Data obsahuje serializovaný JSON objekt obsahující potřebná data. Parametry success a error jsou ukazateli na funkce, které se mají vykonat v případě (ne)úspěšného volání AJAXu. 
		\end{algorithm}	
		
		
	\end{itemize} 
\end{itemize} 

\begin{itemize}
	\item
	\textbf{Component.js}: Jeden ze stavebních kamenů celé aplikace. Představuje objekt obalující všechna zadefinovaná view. Tudíž jakékoliv informace uložené v modelu jsou dostupné napříč celou aplikací. Právě zde se uplatní modely nesoucí znalosti o použitém zařízení klienta nebo sloužící pro překlady textů.
	\item
	\textbf{index.html}: Jedná se o soubor, který je implicitně volán v případě, že uživatel ve svém webovém prohlížeči zadá adresu, pod kterou se požadovaná aplikace skrývá. Nacházejí se zde základní informace nutné pro správné spuštění aplikace. Nejdůležitější z nich je zadefinování zdrojů frameworku SAPUI5. To spočívá v odkazu na obsáhlý soubor sap-ui-core.js (dále již jen jádro), které představuje kostru nutnou pro běh aplikace. Použití slova core - jádro v názvu pochopitelně není náhoda. Jádro si v průběhu používání aplikace dotahuje další frameworkové knihovny jako jsou například komponenty pro uživatelské vstupní pole typu datum, kdy je uživateli poskytnuto příjemné kalendářní rozhraní pro výběr požadovaného data. Tyto knihovny jsou stahovány až v případě, že si je uživatelova interakce vyžaduje. Dochází tedy k tak zvanému lazy loadingu. Stanovit přístup k jádru se dá pomocí dvou základních metod. První z nich je mapování na veřejně dostupný soubor pod adresou \url{https://sapui5.hana.ondemand.com/resources/sap-ui-core.js}. Druhým způsobem je mít veškeré knihovny dostupné v run-timeovém prostředí aplikace. Zároveň se také jedná o možnost použitou v této implementaci. Interní bezpečností politika totiž nedovoluje volání mimo vnitřní síť. Dále se zde nacházejí odkazy na externí knihovny nutné pro správný chod aplikace v rámci prostředí používaném uživatelem. V rámci hierarchie HTML je vytvořen element body, do kterého je vložena celá instance vytvořené aplikace odvozené od staženého jádra. Jedná se zpravidla o jediný kus čistého HTML při tvorbě aplikace ve frameworku SAPUI5.
	\paragraph{Aplikační Cache} Implicitně jsou veškeré zdroje a knihovny používané ve frameworku ukládány do mezipaměti prohlížeče, aby byla uživateli zkrácena doba načítání a nedocházelo k opakovanému stahování potřebných souborů. To s sebou ovšem přináší problém při vydávání nové verze aplikace. V případě, že by došlo ke změně některého ze souborů a libovolný uživatel měl uloženy staré soubory v mezipaměti, pravděpodobně by se stala aplikace nefunkční. Tento problém je oficiálně řešen na straně SAP Gateway, kde dochází k porovnání jednotlivých soborů před samotným stažením. Tato funkcionalita bohužel není implicitně ve frameworku zanesena, nicméně existují mechanismy, které obdobné chování umožňují. Celý proces spočívá v zadefinování ResourceServletu, který porovnává datum poslední modifikace souboru s tím, který má uživatel uložen v mezipaměti. Tento Servlet musí být zadefinován v souboru web.xml a aplikace musí dostat informaci o tom, u kterých souborů má k takové kontrole docházet. Proto je v tomto HTML souboru odkaz na soubor \textbf{sap-ui-cachebuster-info.json}, obsahující JSON pole, s dvěma prvky. Jedním z nich je relativní cesta k požadovanému souboru a druhým je datum poslední modifikaci. Pro správný chod aplikace je tak nutné při každém exportování dbát na aktualizaci potřebných záznamů.
	\item
	\textbf{login.html}: V rámci JAAS logiky pro autorizaci a autentifikaci je definována implicitní HTML stránka, která je vyvolána v případě, že uživatel nemá doposud vytvořenou vůči aplikaci session. Z důvodu zachování konzistetního vzhledu aplikací je i pro přihlášení vytvořena SAPUI5 aplikace. Umístit ji zde v rámci jednoho projektu by způsobilo značnou nepřehlednost a taktéž by mohlo způsobit neočekáváné problémy při vývoji. Z toho důvodu má HTML stránka login.html velmi jednoduchou funkčnost. A tou je přesměrování uživatele do přihlašovací SAPUI5 aplikace.
	\item
	\textbf{manifest.json}: Manifest slouží především k rychlému definování možného směrování v rámci aplikace. Dochází zde k mapování uživatelovi aktuální url na jednotlivá view. Lze tak například přiřadit k postfixu pm/\#/vyroba view \uv{Vyroba} a tím tak uživateli zobrazit očekávané informace. Dále se zde dají zadefinovat modely přiřazené komponentě nebo zdroje kaskádových stylů. 
\end{itemize} 

\subsection{Jednotlivé stránky aplikace}
V této kapitole jsou popsány a zobrazeny nejdůležitější stránky aplikace. 

\begin{figure}[H]
	\centering
	\includegraphics[]{images/fiori_app_view_struct}
	\caption{Diagram případu užití pro správu poruch}
	\label{img:views_strucutre}
\end{figure}

\begin{figure}[H]
	\centering
	\includegraphics[]{images/fiori_app_controller_struct}
	\caption{Diagram případu užití pro správu poruch}
	\label{img:controlllers_structure}
\end{figure}

\subsubsection{App}
\label{sssec:fiori_app}
Není ani tak stránkou jako spíš základním view aplikace. V podstatě pouze všechny ostatní view obaluje a svůj obsah dynamicky střídá na základě uživatelových kroků. Je nastaveno jako výchozí pro všechny možné kombinace URL, které je uživatel v rámci aplikace schopen vytvořit.
\paragraph{View} View je zadefinováno způsobem zobrazeném v následujícím kódu \ref{code:view_app_definition}.
\begin{algorithm}[H]
	\begin{lstlisting}[language=xml]      
<mvc:View controllerName="sap.ui.mo.pm.controller.App"
          xmlns="sap.m" xmlns:mvc="sap.ui.core.mvc" >
  <App id="app" />
</mvc:View>
	\end{lstlisting}
	\caption{XML definice view App}	
	\label{code:view_app_definition}
\end{algorithm}	
Definice neříká nic jiného, než ze v případě zobrazení tohoto view má být vnořena standardní SAPUI5 komponentu App, která slouží jako základ aplikace. V rámci této komponenty jsou následně agregována view popsaná dále. Jak je v kódu vidět, je zde parametr \uv{controllerName}, který přiřazuje k view controller. 
\paragraph{Controller} Stejně jako v ostatních případech je z důvodu přehlednosti totožně pojmenován. Jedinou jeho funkčností je v první fázi spouštění apliakce nastavit dále neměnně atributy. Jak je vidět v ukázce kódu \ref{code:controller_app_definition} níže, dochází zde k nastavení výchozího jazyka aplikace a kaskádových stylů v závislosti na typu použitého zařízení. Jiné styly jsou tak použity pro zařízení s rozlišením odpovídající tabletům, mobilů, nebo desktopům. V potaz se bere i dotykový displej, vyžadující si například větší tlačítka než by tomu bylo v případě použití obyčejného monitoru.
\begin{algorithm}[H]
	\begin{lstlisting}[language=java]      
BaseController.extend("sap.ui.pm.controller.App",{
  onInit : function() {
    sap.ui.getCore().getConfiguration().setLanguage("cs");
    var component = this.getOwnerComponent();
    var css = component.getContentDensityClass();
    this.getView().addStyleClass(css);
  }
}; 
	\end{lstlisting}
	\caption{XML definice view App}	
	\label{code:controller_app_definition}
\end{algorithm}	
Za zmínku ovšem stojí i první řádek zobrazeného kódu \ref{code:controller_app_definition} výše. Výraz \uv{BaseController.extend("sap.ui.mo.pm.controller.App"} říká, že nedochází k rozšíření standardního controlleru frameworku, ale v tomto případě k aplikaci přiloženému controlleru pojmenovaného \uv{BaseController}.
\paragraph{BaseController} Od tohoto controlleru odvozuji všechny ostatní implementované. To protože se nemalá část funkcí dá využít na více místech v aplikaci a nemusí tak docházet k duplikování kódu, které by přinášelo riziko snížení konzistence a zvýšení pracnosti v případě budoucích změn nebo opravování chyb. Tento controller je již odvozen od standardního frameworkového controlleru \uv{sap.ui.core.mvc.Controller}. 
Následující ukázka kódu \ref{code:controller_base_definition} zobrazuje tři takové společné funkce. Celý výčet je pochopitelně mnohem delší, ale tyto  byly vybrány, protože jsou nejčastěji používané a poslouží tak k dobré demonstraci snížené pracnosti a náročnosti na údržbu aplikace z pohledu vývojáře. Funkce zobrazené v ukázce kódu \ref{code:controller_base_definition} jsou posléze krátce popsány.
\begin{algorithm}[H]
	\begin{lstlisting}[language=java]      
Controller.extend("sap.ui.pm.controller.BaseController",{

  createDialog : function(that, id, dialog, fragment) {
    if (!dialog) {
      var frag = sap.ui.xmlfragment(id, fragment, that);
      that.getView().addDependent(frag);
      jQuery.sap.syncStyleClass("sapUiSizeCompact", 
                                that.getView(), res);
      return frag;
    }
    return dialog;
  },  
  
  setModel : function(oModel, sName) {
    return this.getView().setModel(oModel, sName);
  },

  getI18NText : function(that, id) {
    var oModel = that.getView().getModel("i18n");
    var oBundle = oModel.getResourceBundle();
    return oBundle.getText(id);
  }, 
}; 
	\end{lstlisting}
	\caption{XML definice view App}	
	\label{code:controller_base_definition}
\end{algorithm}	

\begin{itemize}
	\item
	\textbf{createDialog}: Aby mohly být v rámci controlleru vytvářeny jednotlivé dialogy (až desítky na jeden controller) téměř bezpracně, byla vytvořena následující prototypová funkce vytvářející v požadovaném controlleru objekt reprezentující dialog. K vytvoření takového dialogu poté stačí jeden příkaz jako v ukázce \ref{code:create_dialog} níže. 
\begin{algorithm}[H]
	\begin{lstlisting}[language=java]      
that.oEqunrSTD = that.createDialog(that, that.oEqunrSTD,
   "../view.dialogs.EqunrSelectTreeDialog");
	\end{lstlisting}
	\caption{Ukázka kódu pro vytvoření objektu dialogu}	
	\label{code:create_dialog}
\end{algorithm}		
	\item
	\textbf{setModel}: Slouží k provázání modelu (objekt obsahující data) s požadovaným view. Každá UI komponenta frameworku umožňuje svázání s modelem a jeho konkrétním prvkem. V případě přeřazení takového modelu k view pak může dojít k zobrazení požadované hodnoty. Jelikož svazování dat v modelu s view patří k velmi častým operacím a vyžaduje volání více funkcí, je obaleno do této metody, které stačí předat jako parametr požadovaný model a jeho jméno.
	\item
	\textbf{getI18NText}: Velmi často v controlleru dojde k situaci, že je potřeba uživateli sdělit nějakou informaci. Může se jednat například o dialogové okno nebo jenom probliknutí textu informujícího o provedené nějaké akce. Aby v controlleru nebyly ošetřovány situace aktuálního jazyka a podobně, je vytvořena tato funkce, která načte hodnotu prvku v aktuálním používaném jazyce z příslušného internacionalizačního souboru. 
\end{itemize}	

\subsubsection{PM - Úvodní stránka}
\label{sssec:fiori_pm}
Reprezentuje úvodní obrazovku, jejíž obsah je přímo závislý na rolích, které má uživatel k dispozici. Zde bude uživateli kromě otevření aplikací umožněno změnit si heslo.
\paragraph{View}
Pro demonstraci struktury jednotlivých view je v této, jakožto první, ukázce obsahující větší počet komponent zobrazena téměř celá XML struktura rozdělena do tří bloků. V implementaci 
\begin{algorithm}[H]
	\begin{lstlisting}[language=xml]      
<Page>
  <customHeader>
    <Bar design="Header">
      <contentLeft />
      <contentMiddle>
        <Title text="{i18n>pmPageTitle}"  />
      </contentMiddle>
      <contentRight>
        <Button icon="://key" press="onEditPassword" />
        <Button icon="://log" press="onLogout" />
      </contentRight>
    </Bar>
  </customHeader>
	\end{lstlisting}
	\caption{XML definice hlavičky úvodní stránky}	
	\label{code:view_pm_header_definition}
	\small Ve všech aplikacích se jedná o lehkou modifikaci této struktury. Zpravidla dochází ke změnám pouze u svazování nadpisu a použitých tlačítkách v horní liště. Jak lze vykoukat, hlavičková lišta je rozdělena do tří částí, které lehce umožňují zarovnání použitých komponent.
\end{algorithm}	
\begin{algorithm}[H]
	\begin{lstlisting}[language=xml]      
  <VBox width="100%" justifyContent="Center" >
    <l:Grid id="gridContainer" defaultSpan="L3 M6 S6" />
  </VBox>
	\end{lstlisting}
	\caption{XML definice obsahu úvodní stránky}	
	\label{code:view_pm_content_definition}
	\small Jak je v ukázce vidět, jedná se o velmi jednoduchý obsah stránky. V podstatě je pouze zadefinováno rozložení Grid kontejneru, které má implicitně responzivní chování. Vývojář tedy nemusí nijak extra řešit počet zobrazených agregovaných komponent, framework to zvládá sám.  
\end{algorithm}	
\begin{algorithm}[H]
	\begin{lstlisting}[language=xml]      
</Page>
	\end{lstlisting}
	\caption{XML definice zápatí úvodní stránky}	
	\label{code:view_pm_footer_definition}
	\small V tomto případě se jedná pouze o ukončení stránky. U ostatních aplikací mohou být zobrazovány například různá zápatí obsahující tlačítka s funkcemi a podobně.
\end{algorithm}	
\paragraph{Controller} Má v rámci této stránky primární úkol k přidělené požadovaných dlaždic uživateli v závislosti na přidělených rolích. To spočívá v načtení uživatelských rolí a přidání dlaždic do Grid kontejneru zadefinovaného v ukázce kódu \ref{code:view_pm_content_definition} výše. 
\paragraph{Stránka} Výsledná podoba stránky je zobrazena na obrázku \ref{img:view_pm} níže. Jedná se o velmi minimalistické provedení, jelikož se neočekává, že by zde uživatel trávil větší množství času. Obrázek je rozdělen na dvě části. V první je podoba stránky zobrazené v běžném desktop rozlišení (více jak 1200 pixelů na šířku). V druhé části je zobrazena podoba v mobilním zařízení odpovídajícímu dnešnímu standardnímu chytrému telefonu s obrazovkou velkou přibližně 5 palců (šířka alespoň 350 pixelů na šířku).
\begin{figure}[H]
	\centering
	\includegraphics[width=1\textwidth]{images/view_pm}
	\caption{Úvodní stránka PM SAPUI5 aplikace}
	\label{img:view_pm}
	\small Podoba stránky odpovídá případu, kdy má uživatel přiřazené všechny dosavadní role.
\end{figure}

\subsubsection{Pozadavek - Založení požadavku na údržbu}
\label{sssec:pozadavek}
Stránka je navržena tak, aby odpovídala funkčnímu požadavku \ref{sssec:fc_zalozeni_pozadavku}. Je zde proto vytvořen formulář umožňující zadat veškeré potřebné údaje ke specifikování požadavku. 
\paragraph{View}
Jelikož se jedná v podstatě pouze o formulář určený k vyplnění od uživatele, je obsahem view především komponenta SimpleForm, pomocí které lze snadno vytvořit cílený formulář. 
\begin{algorithm}[H]
	\begin{lstlisting}[language=xml]      
<f:SimpleForm>
  <Label text="{i18n>pozadavekTplnrLabel}" />
  <Input value="{hlaseni>/tplnr}" showValueHelp="true" 
         valueHelpRequest="onTplnrMatchCodeRequest" />
  <ndc:BarcodeScannerButton scan="handleScan" />
  <Label text="{i18n>pozadavekVybaveniLabel}" />
  <Text text="{hlaseni>/eqktx}" />
</f:SimpleForm>
	\end{lstlisting}
	\caption{XML definice obsahu úvodní stránky}	
	\label{code:view_pozadavek_form}
	\small Komponenta SimpleForm vytváří formulář na základě komponenty Label, která od sebe jednotlivé části separuje. Všechny elementy od Labelu až k dalšímu tvoří jeden celek a jsou výsledně rendrovány v jedné skupině.
\end{algorithm}	
\paragraph{Controller}
Úkolem controlleru na této stránce je především předvyplnění uživatelských atributů do vstupního formuláře a jeho následné odeslání na backend. To spočívá v načtení uživatelských dat v případě navštívení stránky s následným  naplnění modelu příslušnými daty. Obsahuje také funkce pomáhající uživateli vybrat data. Takovým příkladem může být načtení hierarchie technických míst. K tomu je zapotřebí poslat požadavek pro data. K tomu poslouží funkce query implementující AJAX request \ref{code:ajax} v JavaScriptové knihovně utils. Předáním lokálních funkcí pro úspěšné a neúspěšné volání poté může dojít zpracování přijatých dat nebo chyb. Pomocí funkce setModel popsané v BaseControlleru \ref{code:controller_base_definition} a nebo přístupem ke konkrétnímu prvku modelu (funkce setProperty) lze potom nastavit požadovaná data do svázaného formuláře. 
\paragraph{Stránka}
Výsledná stránka je reprezentována formulářem o sedmi prvcích. Podle typu zadávaných hodnot v poli je pak přizpůsobena komponenta ulehčující uživateli vyplnění. Pro technické místo je možné si nechat zobrazit dialog s hierarchií technických míst a vybavení si z něho vybrat. U elementů s pevně daným výběrem a zároveň striktně omezeným počtem je pak vybrána komponenta SelectList a podobně. Podoba stránky pro desktop a mobilní zařízení je vidět na obrázku \ref{img:view_zalozeni_pu} níže.
\begin{figure}[H]
	\centering
	\includegraphics[width=1\textwidth]{images/view_zalozeni_pu}
	\caption{Stránka pro založení požadavku na údržbu}
	\label{img:view_zalozeni_pu}
\end{figure}

\subsubsection{Vyroba - Operátor výroby}
\label{sssec:fiori_vyroba}
Stránka je navržena tak, aby odpovídala funkčním požadavkům spadající pod roli Operátora výroby. Musí zde tedy býti dostupný seznam jednotlivých hlášení (poruchy, prevence a údržby) pro jeho přidělené pracoviště. V rámci jednotlivých hlášení je zapotřebí mít k dispozici relevantní operace, které s nimi operátor může provést.
\paragraph{View}
Z důvodu responzibility se jedná o první stránku, kde je zapotřebí již v rámci view řešit rozlišení používaného zařízení z důvodu obsáhlé tabulky, která se na mobilních zařízeních nebude vhodně zobrazovat. V horní části obrazovky jsou navrženy dva filtry. První z nich je na typ zobrazovaného hlášení a druhým je filtr na technická místa. Filtr přes druh hlášení nejen, že filtruje data, ale i mění zobrazované informace. Požadovaná zobrazovaná data k jednotlivých typům hlášením nejsou totiž identická. Tato část je pro desktopová i mobilní zařízeni stejná. Dále se však struktura stránky liší. 
\paragraph{View (desktop)}
Pod filtry je navržena tabulka se všemi možnými sloupci, které se mohou v rámci stránky zobrazit. V rámci jednotlivých sloupců jsou pak přiděleny agregace na UI komponenty vyhovujícím požadavkům. Jedná se především o texty a tlačítka, která jsou zobrazována v závislosti na stavu (statusu) hlášení. O logiku se stará knihovna utils \ref{file:utils.js} implementující funkce rozhodující o informaci zobrazit nebo nezobrazit.
\paragraph{View (mobile)}
Pod filtry je navržen komplikovanější rozložení skládající se z desítek komponent různých layoutů uspořádaných do hierarchické struktury. Bylo zapotřebí změnit celou strukturu oproti tabulkovému zobrazení. Každé hlášení tak horizontálně zabírá více místa. Namísto maximálně dvou řádků v rámci jednoho hlášení se tak nyní vyskytuje až sedm řádků informací.
\paragraph{Controller}
Kromě standardních činností, jako je odchytávání uživatelovi interakci s patřičným zpracováním, je zde zapotřebí dynamicky řešit obsah hlavního view. Toho je dosaženo za pomocí modelu zařízení a \textbf{návrhového vzoru pozorovatel}, který řeší informování požadovaných objektů o změně stavu jiného objektu. Pozorovaným objektem je v tomto případě model zařízení (konkrétně jeho část řešící aktuální rozlišení). Pozorovatelem je objekt (funkce) controlleru. 
\begin{algorithm}[H]
	\begin{lstlisting}[language=java]     
var dev = this.getOwnerComponent().getModel("device");
dev.getProperty("/resize").attachHandler(this.onResize);
	\end{lstlisting}
	\caption{Přiřazení posluchače ve formě funkce k hodnotě modelu}	
	\label{code:resize_attach_handler}
	\small Jedná se o výtažek kódu v inicializační funkci onInit. Spočívá v načtení device modelu a přiřazení posluchače v podobě funkce onResize, která se provede vždy při změně rozlišení. To znamená zahrnuje i případ otočení displeje na mobilních zařízeních.
\end{algorithm}	
\begin{algorithm}[H]
	\begin{lstlisting}[language=java]      
onResize : function(window) {
  var layout = that.getView().byId("notifLayout");
  layout.removeAllContent();
  if (window.width > 1024) {
    layout.addContent(that.notifTable);
  } else {
    layout.addContent(that.notifList);
}
	\end{lstlisting}
	\caption{Implementace funkce onResize}
	\label{code:resize_handler}
	\small Jako hraniční hodnota pro zobrazení tabulky nebo listu byla vybrána hodnota 1200 pixelů. V případě změny dojde k odebrání fragmentu z layoutu a přiřazení adekvátního obsahu.
\end{algorithm}	
\paragraph{Stránka}
Cílem návrhu této stránky bylo maximální možné eliminování tlačítek a textů, které uživatel nepotřebuje znát. Tudíž všechny texty i tlačítka se zobrazují v případě, že mají nějaký význam. U textů to jsou pro uživatele potřebné informace k vykonávání jeho práce. V případě tlačítek umožňujících akce nad daným hlášením je tento problém řešen zobrazením jenom těch tlačítek, které lze nad daným hlášením v danou chvíli provést. Zrušit hlášení tak lze pouze do doby, než s ním někdo začne pracovat. Zobrazovat texty k hlášení jdou pouze tehdy, když už nějaký text k němu existuje a podobně. Podoba stránky pro desktop a mobilní zařízení je vidět na obrázku \ref{img:view_vyroba} níže.
\begin{figure}[H]
	\centering
	\includegraphics[width=1\textwidth]{images/view_vyroba}
	\caption{Stránka pro operátora údržby}
	\label{img:view_vyroba}
\end{figure}

\subsubsection{Udrzba - Údržbář}
Stránka je navržena tak, aby odpovídala funkčním požadavkům spadající pod roli Údržbáře. Musí zde tedy býti dostupný seznam jednotlivých hlášení (poruchy, prevence a údržby) připravených k provedení servisního úkonu. V rámci jednotlivých hlášení je zapotřebí mít k dispozici relevantní operace, které s nimi údržbář může provést.
\paragraph{View}

\paragraph{View (mobile)}
\paragraph{View (desktop)}
\paragraph{Controller}
\paragraph{Stránka}
\begin{figure}[H]
	\centering
	\includegraphics[width=1\textwidth]{images/view_udrzba}
	\caption{Diagram případu užití pro správu poruch}
	\label{img:view_udrzba}
\end{figure}	

\subsubsection{Administrace}
\paragraph{View}
\paragraph{Controller}
\paragraph{Stránka}
\begin{figure}[H]
	\centering
	\includegraphics[width=1\textwidth]{images/view_administrace}
	\caption{Diagram případu užití pro správu poruch}
	\label{img:view_administrace}
\end{figure}	
\begin{figure}[H]
	\centering
	\includegraphics[width=0.6\textwidth]{images/view_administrace_mob}
	\caption{Diagram případu užití pro správu poruch}
	\label{img:view_administrace_mob}
\end{figure}	

\subsubsection{Pomocné dialogy}
\begin{figure}[H]
	\centering
	\includegraphics[width=1\textwidth]{images/view_dialog}
	\caption{Diagram případu užití pro správu poruch}
	\label{img:view_dialog}
\end{figure}	

\section{LOGIN SAPUI5 Aplikace}

\section{Porovnání vývojových prostředí}

\subsection{Eclipse s pluginem pro SAPUI5}

\subsection{SAP Web IDE}

\section{Testování}

\subsection{Middleware}

\subsection{Frontend}

\section{Doporučení pro vývoj}

\begin{conclusion}
	%sem napište závěr Vaší práce
\end{conclusion}

\bibliographystyle{csn690}
\bibliography{mybibliographyfile}

\appendix

\chapter{Seznam použitých zkratek}
% \printglossaries
\begin{description}
	\item[GUI] Graphical user interface
	\item[XML] Extensible markup language
\end{description}


% % % % % % % % % % % % % % % % % % % % % % % % % % % % 
% % Tuto kapitolu z výsledné práce ODSTRAŇTE.
% % % % % % % % % % % % % % % % % % % % % % % % % % % % 
% 
% \chapter{Návod k~použití této šablony}
% 
% Tento dokument slouží jako základ pro napsání závěrečné práce na Fakultě informačních technologií ČVUT v~Praze.
% 
% \section{Výběr základu}
% 
% Vyberte si šablonu podle druhu práce (bakalářská, diplomová), jazyka (čeština, angličtina) a kódování (ASCII, \mbox{UTF-8}, \mbox{ISO-8859-2} neboli latin2 a nebo \mbox{Windows-1250}). 
% 
% V~české variantě naleznete šablony v~souborech pojmenovaných ve formátu práce\_kódování.tex. Typ může být:
% \begin{description}
% 	\item[BP] bakalářská práce,
% 	\item[DP] diplomová (magisterská) práce.
% \end{description}
% Kódování, ve kterém chcete psát, může být:
% \begin{description}
% 	\item[UTF-8] kódování Unicode,
% 	\item[ISO-8859-2] latin2,
% 	\item[Windows-1250] znaková sada 1250 Windows.
% \end{description}
% V~případě nejistoty ohledně kódování doporučujeme následující postup:
% \begin{enumerate}
% 	\item Otevřete šablony pro kódování UTF-8 v~editoru prostého textu, který chcete pro psaní práce použít -- pokud můžete texty s~diakritikou normálně přečíst, použijte tuto šablonu.
% 	\item V~opačném případě postupujte dále podle toho, jaký operační systém používáte:
% 	\begin{itemize}
% 		\item v~případě Windows použijte šablonu pro kódování \mbox{Windows-1250},
% 		\item jinak zkuste použít šablonu pro kódování \mbox{ISO-8859-2}.
% 	\end{itemize}
% \end{enumerate}
% 
% 
% V~anglické variantě jsou šablony pojmenované podle typu práce, možnosti jsou:
% \begin{description}
% 	\item[bachelors] bakalářská práce,
% 	\item[masters] diplomová (magisterská) práce.
% \end{description}
% 
% \section{Použití šablony}
% 
% Šablona je určena pro zpracování systémem \LaTeXe{}. Text je možné psát v~textovém editoru jako prostý text, lze však také využít specializovaný editor pro \LaTeX{}, např. Kile.
% 
% Pro získání tisknutelného výstupu z~takto vytvořeného souboru použijte příkaz \verb|pdflatex|, kterému předáte cestu k~souboru jako parametr. Vhodný editor pro \LaTeX{} toto udělá za Vás. \verb|pdfcslatex| ani \verb|cslatex| \emph{nebudou} s~těmito šablonami fungovat.
% 
% Více informací o~použití systému \LaTeX{} najdete např. v~\cite{wikilatex}.
% 
% \subsection{Typografie}
% 
% Při psaní dodržujte typografické konvence zvoleného jazyka. České \uv{uvozovky} zapisujte použitím příkazu \verb|\uv|, kterému v~parametru předáte text, jenž má být v~uvozovkách. Anglické otevírací uvozovky se v~\LaTeX{}u zadávají jako dva zpětné apostrofy, uzavírací uvozovky jako dva apostrofy. Často chybně uváděný symbol "{} (palce) nemá s~uvozovkami nic společného.
% 
% Dále je třeba zabránit zalomení řádky mezi některými slovy, v~češtině např. za jednopísmennými předložkami a spojkami (vyjma \uv{a}). To docílíte vložením pružné nezalomitelné mezery -- znakem \texttt{\textasciitilde}. V~tomto případě to není třeba dělat ručně, lze použít program \verb|vlna|.
% 
% Více o~typografii viz \cite{kobltypo}.
% 
% \subsection{Obrázky}
% 
% Pro umožnění vkládání obrázků je vhodné použít balíček \verb|graphicx|, samotné vložení se provede příkazem \verb|\includegraphics|. Takto je možné vkládat obrázky ve formátu PDF, PNG a JPEG jestliže používáte pdf\LaTeX{} nebo ve formátu EPS jestliže používáte \LaTeX{}. Doporučujeme preferovat vektorové obrázky před rastrovými (vyjma fotografií).
% 
% \subsubsection{Získání vhodného formátu}
% 
% Pro získání vektorových formátů PDF nebo EPS z~jiných lze použít některý z~vektorových grafických editorů. Pro převod rastrového obrázku na vektorový lze použít rasterizaci, kterou mnohé editory zvládají (např. Inkscape). Pro konverze lze použít též nástroje pro dávkové zpracování běžně dodávané s~\LaTeX{}em, např. \verb|epstopdf|.
% 
% \subsubsection{Plovoucí prostředí}
% 
% Příkazem \verb|\includegraphics| lze obrázky vkládat přímo, doporučujeme však použít plovoucí prostředí, konkrétně \verb|figure|. Například obrázek \ref{fig:float} byl vložen tímto způsobem. Vůbec přitom nevadí, když je obrázek umístěn jinde, než bylo původně zamýšleno -- je tomu tak hlavně kvůli dodržení typografických konvencí. Namísto vynucování konkrétní pozice obrázku doporučujeme používat odkazování z~textu (dvojice příkazů \verb|\label| a \verb|\ref|).
% 
% \begin{figure}\centering
% 	\includegraphics[width=0.5\textwidth, angle=30]{cvut-logo-bw}
% 	\caption[Příklad obrázku]{Ukázkový obrázek v~plovoucím prostředí}\label{fig:float}
% \end{figure}
% 
% \subsubsection{Verze obrázků}
% 
% % Gnuplot BW i barevně
% Může se hodit mít více verzí stejného obrázku, např. pro barevný či černobílý tisk a nebo pro prezentaci. S~pomocí některých nástrojů na generování grafiky je to snadné.
% 
% Máte-li například graf vytvořený v programu Gnuplot, můžete jeho černobílou variantu (viz obr. \ref{fig:gnuplot-bw}) vytvořit parametrem \verb|monochrome dashed| příkazu \verb|set term|. Barevnou variantu (viz obr. \ref{fig:gnuplot-col}) vhodnou na prezentace lze vytvořit parametrem \verb|colour solid|.
% 
% \begin{figure}\centering
% 	\includegraphics{gnuplot-bw}
% 	\caption{Černobílá varianta obrázku generovaného programem Gnuplot}\label{fig:gnuplot-bw}
% \end{figure}
% 
% \begin{figure}\centering
% 	\includegraphics{gnuplot-col}
% 	\caption{Barevná varianta obrázku generovaného programem Gnuplot}\label{fig:gnuplot-col}
% \end{figure}
% 
% 
% \subsection{Tabulky}
% 
% Tabulky lze zadávat různě, např. v~prostředí \verb|tabular|, avšak pro jejich vkládání platí to samé, co pro obrázky -- použijte plovoucí prostředí, v~tomto případě \verb|table|. Například tabulka \ref{tab:matematika} byla vložena tímto způsobem.
% 
% \begin{table}\centering
% 	\caption[Příklad tabulky]{Zadávání matematiky}\label{tab:matematika}
% 	\begin{tabular}{|l|l|c|c|}\hline
% 		Typ		& Prostředí		& \LaTeX{}ovská zkratka	& \TeX{}ovská zkratka	\tabularnewline \hline \hline
% 		Text		& \verb|math|		& \verb|\(...\)|	& \verb|$...$|		\tabularnewline \hline
% 		Displayed	& \verb|displaymath|	& \verb|\[...\]|	& \verb|$$...$$|	\tabularnewline \hline
% 	\end{tabular}
% \end{table}
% 
% % % % % % % % % % % % % % % % % % % % % % % % % % % % 

\chapter{Obsah přiloženého CD}

%upravte podle skutecnosti

\begin{figure}
	\dirtree{%
		.1 readme.txt\DTcomment{stručný popis obsahu CD}.
		.1 exe\DTcomment{adresář se spustitelnou formou implementace}.
		.1 src.
		.2 impl\DTcomment{zdrojové kódy implementace}.
		.2 thesis\DTcomment{zdrojová forma práce ve formátu \LaTeX{}}.
		.1 text\DTcomment{text práce}.
		.2 thesis.pdf\DTcomment{text práce ve formátu PDF}.
		.2 thesis.ps\DTcomment{text práce ve formátu PS}.
	}
\end{figure}

\end{document}
